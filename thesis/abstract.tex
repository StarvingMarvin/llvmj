\begin{abstract}

Programski prevodioci više od pedeset godina predstavljaju predmet mnogih akademskih radova i implementacija dokazanih u praksi.
Uprkos tome nova istraživanja donose dalja unapređenja, a moć savremenih računara čini neke tehnike praktičnijim.

Na primeru jednostavnog jezika (MicroJava) biće predstavljena upotreba dva alata proistekla iz akademskih radova koji imaju široku primenu i bitno olakšavaju implementaciju programskih prevodioca.

Prvi alat je ANTLR parser generator. Karakteristike ANTLR-a omogućavaju jednostavnu leksičku, sintaksnu i semantičku analizu računarskih jezika.

Drugi alat je LLVM, koji predstavlja bogatu infrastrukturu za implementaciju optimizovanih programskih prevodioca za veliki broj procesorskih arhitektura pod POSIX kompatabilnim sistemima.

\end{abstract}
