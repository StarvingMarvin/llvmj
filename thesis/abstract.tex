\begin{abstract}

Programski prevodioci već više od pedeset godina predstavljaju predmet mnogih akademskih radova i kroz primenu potvrđenih implementacija. Uprkos tome nova istraživanja donose dalja unapređenja, a savremeni računari čine mnoge stvari praktičnijima nego ranije.

Na primeru jednostavnog jezika (MicroJava) nameravam da pokažem upotrebu dva alata proistekla iz akademskih radova koji imaju široku primenu i bitno olakšavaju implementaciju programskih prevodioca.

Prvi alat je ANTLR koji pruža pogodan skup alata pogodnih za sintaksnu i semantičku analizu računarskih jezika.

Drugi alat je LLVM. On predstavlja bogatu infrastrukturu za implementaciju optimizovanih programskih prevodioca za veliki broj procesorskih arhitektura pod POSIX kompatabilnim sistemima.

\end{abstract}
