\chapter{Zaključak}
\label{ch:zakljucak}

Generisanje optimizovanog koda koji će da radi na više različitih procesorskih arhitektura i operativnih sistema je obiman posao koji je često morao biti ponovljan za svaki novi programski jezik ili novu implementaciju postojecih jezika.
U poslednjih nekoliko godina prisustvovali smo povećanom interesovanju za dizajniranje novih jezika, pa ga tako Rust, Swift i Julia koriste u svojoj implementaciji.
Takođe je postojanje LLVM-a dovelo do pravljenja novih implemendacija postojećih jezika kao što je pre svega Clang projekat koji implementira C, C++ i Objective C komplajlere.

U poređenju sa generisanjem koda, parsiranje predstavlja manji izazov.
Sve jedno u zavisnosti od potreba i preferenci, treba odabrati odgovarajući projekat koji će taj proces olakšati.
Jedan od ciljeva koji sam želeo da ostvarim u procesu parsiranja jeste izbegavanje mešanja gramatike i C koda u istom fajlu.
Iako ANTLR omogućava ovakav način rada, uz AST gramatku bilo je moguće skoro potpuno izbeći mešanje.
Jedina mesta gde su upotrebljeni C izrazi su pozivanje \code{SKIP()} funkcije za ignorisanje sadržaja komentara i uklanjanje jednostrukih znakova navoda iz vrednosti neterminala za \code{char} literal.
Drugi način na koji bi bilo moguće izbegnuti mešanje više jezika, jeste pristup koji imaju PEG parseri, gde se celokupan parser definiše pozivanjem odgovarajućih funkcija definisanih u parserskoj biblioteci.

Postoji još dosta mogućnosti LLVM-a koje u ovom projektu nisu iskorišćene.
Moguće je umetati anotacije u bajt kod i koristiti ih za čuvanje podataka o izvornom kodu iz kojeg je program generisan.
Na osnovu anotacija moguće je realizovati profilisanje programa, izvršavanje programa instrukciju po instrukciju i slično.
