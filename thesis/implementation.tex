\chapter{Implementacija MicroJava prevodioca}
\label{ch:implementacija}

Ovo poglavlje opisuje arhitekturu i implementacione detalje. Posebna pažnja je posvećena implementaciji semantike jezika.

\section{Parser}

Ako se uporede BNF gramatika iz MicroJava specifikacije i listing ANTLR gramatike, vidi se da je bilo moguće preslikati sintaksna pravila bez većih izmena.
Glavni dodatak u ANTLR gramatici jesu izrazi za konstrukciju sintaksnog stabla.
Primer programa i generisanog sintaksnog stabla nalazi se u dodatku \ref{ch:example}.

U slučaju sintaksne greške, čvor će biti označen kao nevalidan, ali će ostatak stabla biti dostupan, tako da je moguće izvršiti dodatnu analizu validnog koda.

Kako ANTLR generiše C kod, rezultat parsiranja programa je struktura \mj{ANTLR3\_BASE\_TREE\_struct} koja predstavlja sintaksno stablo.
Struktura sadrži podatke kao što su tip tokena trenutnog čvora, tekst i lokacija izvornog koda koja odgovara čvoru, roditelja i listu dece.
Tip tokena je celobrojna konstanta čiju vrednost dodeljuje ANTLR prilikom generisanja koda.
Za jednostavniji rad  u C++u sa dobijenom strukturom implementirane su
\mj{AstWalker}, \mj{NodeVisitor} i \mj{nodeiterator} klase i nekoliko pomoćnih funkcija.

\begin{figure}[h]
	\centering
	\begin{tikzpicture}
	\tikzumlset{font=\footnotesize}

	\begin{umlpackage}[x=0,y=0]{parser}

		\umlclass[x=20pt,y=20pt]{AstWalker}
			{
				- stack: vector$\langle$AST$\rangle$		
			}{
	            + addVisitor(tokenType: uint32\_t, visitor: NodeVisitor): void \\
	            + getVisitor(uint32\_t tokenType): NodeVisitor \\
	            + visit(AST ast): void \\
	            + tokenType(): uint32\_t \\
	            + tokenText(): char$\ast$ \\
	            + childCount(): size\_t \\
	            + nilNode(): bool \\
	            + printPosition(ostream\& os): ostream\& \\
	            + firstChild(): nodeiterator \\
	            + lastChild(): nodeiterator \\
	            + currentNode(): AST \\
	            + template $\langle$class T$\rangle$ setData(T$\ast$ data): void \\
	            + template $\langle$class T$\rangle$ getData(): T$\ast$
			}

		\umlclass[x=20pt,y=200pt]{NodeVisitor}
			{}{
				+ operator()(walker: AstWalker): void
			}
			
		\umlemptyclass[x=250pt,y=20pt]{nodeiterator}
		
		\umlclass[x=250pt,y=190pt]{std::iterator}
			{}{
				+ operator++() \\
			    + operator*() \\
			    ...
			}

		
		\umluniaggreg[attr=defaultVisitor|1, anchor1=54,anchor2=-20]
			{AstWalker}{NodeVisitor}

    	\umluniaggreg[attr=visitors|*,pos=0.5]
			{AstWalker}{NodeVisitor}

		\umldep[anchor1=-160,anchor2=126]
			{NodeVisitor}{AstWalker}
			
		\umlimpl{nodeiterator}{std::iterator}
		
		\umluniassoc{AstWalker}{nodeiterator}		
		
%		\umlVHuniaggreg[attr=type|1,anchor1=-130, anchor2=-16,pos=1.9,align=left]
%			{NamedValue}{Type}
%	
%		\umlHVHuniaggreg[attr=returnType|1, anchor1=10,  anchor2=-16, arm2=40pt, pos=1.4]
%			{MethodType}{Type}
%		\umlHVHuniaggreg[attr=*|arguments, anchor1=-10,  anchor2=-16, arm2=55pt, pos=1.8]
%			{MethodType}{Type}
	\end{umlpackage}
\end{tikzpicture} 
	\caption{Klase koje pružaju pomoć prilikom obilaska sintasnog stabla}
	\label{fig:visitor}
\end{figure}

\mj{AstWalker} prihvata registraciju podrazumevanog posetioca kao i instancu posetioca po tipu AST čvora.
Pozivanjem metode \mj{visit(AST node)}, dati čvor se smešta na stek i poziva se odgovarajući \mj{Visitor} za tip tog noda.
U toku rada, \mj{Visitor} može da poziva metode relativne za trenutni čvor, kao što su \mj{tokenText()}, \mj{firstChild()} ili \mj{lastChild()}.
Po potrebi, za svako dete-čvor može pozivati \mj{visit()} metodu \mj{AstWalker} klase.
Kada se obrada čvora završi, \mj{AstWalker} skida čvor sa steka i vraća kontrolu pozivaocu.

Klasa \mj{nodeiterator} je implementacija \mj{std::iterator} interfejsa.
Inicijalizuje se čvorom sintaksnog stabla i iterira po deci datog čvora. 
Pošto implementira sve metode predviđene \mj{std::random\_access\_iterator\_tag}-om, moguće je iterirati unapred i unazad, kao i zahtevati odredjeno dete-čvor po indeksu.

\section{Semantika}

\subsection*{Simboli}

Centralno mesto u ovoj fazi MicroJava prevodioca zauzima hijerarhija klasa vezanih za simbole, opsege i tabelu simbola.

U MicroJavi možemo izdvojiti nekoliko vrsta simbola, kao što su primitivni tipovi, promenljive, konstante, klase i metode.
Radi što lakše analize, a kasnije i generisanja koda, ovi simboli predstavljeni su hijerarhijom klasa prikazanoj na dijagramu \ref{fig:symbols}.

\begin{figure}[h]
	\centering
	\begin{tikzpicture}
	\tikzumlset{font=\footnotesize}

	\begin{umlpackage}[x=0,y=0]{symbols}

		\umlclass[x=210pt,y=250pt]{Symbol}
			{+ name : string }{} 

		\umlclass[x=90pt,y=165pt]{Type}
			{}{
				+ operator==(t: Type) : bool \\ 
				+ compatible(t: Type) : bool
			}

		\umlemptyclass[x=290pt,y=176pt]{NamedValue}

		\umlemptyclass[x=210pt,y=176pt]{Program}

		\umlemptyclass[x=40pt,y=85pt]{ReferenceType}

		\umlemptyclass[x=10pt,y=20pt]{Array}

		\umlemptyclass[x=70pt,y=20pt]{Class}

		\umlemptyclass[x=140pt,y=85pt]{MethodType}

		\umlemptyclass[x=250pt,y=85pt]{Method}

		\umlclass[x=330pt,y=80pt]{Constant}
			{+ value : int}{}

		\umlVHVinherit[anchor1=north, anchor2=south, arm2=-15pt]
			{Type}{Symbol}

		\umlVHVinherit[anchor1=north, anchor2=south, arm2=-15pt]
			{NamedValue}{Symbol}

		\umlVHVinherit[anchor1=north, anchor2=south, arm2=-15pt]
			{Program}{Symbol}

		\umlVHVinherit[anchor1=north, anchor2=south, arm2=-15pt]
			{ReferenceType}{Type}

		\umlVHVinherit[anchor1=north, anchor2=south, arm2=-15pt]
			{MethodType}{Type}

		\umlVHVinherit[anchor1=north, anchor2=south, arm2=-15pt]
			{Array}{ReferenceType}

		\umlVHVinherit[anchor1=north, anchor2=south, arm2=-15pt]
			{Class}{ReferenceType}

		\umlVHVinherit[anchor1=north, anchor2=south, arm2=-40pt]
			{Constant}{NamedValue}

		\umlVHVinherit[anchor1=north, anchor2=south, arm2=-40pt]
			{Method}{NamedValue}

		\umlVHVuniaggreg[attr=methodType|1, anchor1=-120, anchor2=-60,arm2=-20pt,pos=1.9,align=left]
			{Method}{MethodType}

		\umlVHuniaggreg[attr=type|1,anchor1=-130, anchor2=-16,pos=1.9,align=left]
			{NamedValue}{Type}
	
		\umlHVHuniaggreg[attr=returnType|1, anchor1=10,  anchor2=-16, arm2=40pt, pos=1.4]
			{MethodType}{Type}
		\umlHVHuniaggreg[attr=*|arguments, anchor1=-10,  anchor2=-16, arm2=55pt, pos=1.8]
			{MethodType}{Type}
	\end{umlpackage}
\end{tikzpicture} 
	\caption{Pojednostavljena struktura klasa koje predstavljaju simbole}
	\label{fig:symbols}
\end{figure}

Prilikom analize programa, bitno je da za tipove vrednosti u nekom izrazu utvrdimo da li su jednaki očekivanom tipu, odnosno da li su kompatibilni u slučaju dodele.
Klasa \code{Type} deklariše virtualne metode \code{operator==} i \code{compatible}, a podrazumevana implementacija odgovara semantici primitivnih tipova: tipovi su jednaki ako se isto zovu, a kompatibilni su samo ako su jednaki.

Klasa \code{ReferenceType} prihvata \mj{null} kao kompatibilan tip.

Tipovi nizova i metoda se implicitno registruju u globalnu tabelu simbola prilikom prvog nailaženja u kodu.

MicroJava metod je implementiran kao vrednost koja ima tip.
Tip metoda sastoji se od povratnog tipa i tipa argumenata. 
U fazi semantičke analize, tip se koristi za proveru argumenata i dodelu rezultata promenljivoj prilikom poziva metode.
Kasnije se u fazi generisanja koda ponovo koristi za pravljenje ekvivalentnog LLVM tipa.
Takođe, metode poseduju i dva opsega vidljivosti, za lokalne promenjive i argumente.

\begin{figure}[h]
	\centering
	\begin{tikzpicture}
	\tikzumlset{font=\footnotesize}

	\begin{umlpackage}[x=0,y=0]{semantics}
		\umlclass[x=0pt,y=0pt]{Scope}{+ name : string }{}
	\end{umlpackage}
\end{tikzpicture}
	\caption{Pojednostavljena struktura klasa koje implementiraju i koriste opsege vidljivosti}
	\label{fig:scopes} 
\end{figure}

Opsezi vidljivosti takođe imaju svoju hijerarhiju, prikazanu na slici \ref{fig:scopes}.
Osnovna funkcionalnost, definisanje i pronalaženje simbola, implementirana je u klasi \mj{Scope}.
Globalni opseg, implementiran u klasi \mj{GlobalScope}, sadrži sve predefinisane tipove i simbole, kao i implicitno generisane tipove nizova i potpise metoda.
Globalni opseg takođe sadrži i referencu na objekat tipa \mj{Program} koja predstavlja spoljnu klasu MicroJava programa.
Ulazni fajl je nevalidan, ako do kraja njegove obrade nije definisana tačno jedna instanca klase \mj{Program}.


\subsection*{Semantička analiza}

Koristeći \mj{AstWalker} klasu definisanu u \mj{parser} paketu, za svaki tip tokena iz sintaksnog stabla definisana je \mj{Visitor} klasa koja proverava semantičku ispravnost izraza i sakuplja simbole. 
Svi posetioci poseduju referencu na objekat tipa \mj{Symbols} koji predstavlja fasadu ka konkretnim simbolima i opsezima vidljivosti \ref{fig:symbols-facade}.

U ovom koraku se proveravaju pravila definisana kontekstnim uslovima MicroJava specifikacije:
\begin{itemize}
\item Sva imena su deklarisana pre prvog korišćenja
\item Vrednosti sa leve strane iskaza dodele su promenljive, a ne konstante
\item Tipovi promenljivih ili literala odgovaraju izrazu u kome se nalaze
\item Instrukcija \code{break} se nalazi unutar petlje
\item Funkcija koja vraća vrednost sadrži \code{return} iskaz i vraćena vrednost je ispravnog tipa
\item Postoji metoda \code{main} koja ne prima argumente i ne vraća vrednost
\end{itemize}

Sa uspešno izvršenom proverom ulaznog programa i popunjenom tabelom simbola, sve je spremno za generisanje koda.
\begin{figure}[h]
	\centering
	\begin{tikzpicture}
	\tikzumlset{font=\footnotesize}
	\begin{umlpackage}[x=-2.5,y=0]{semantycs}
		\umlemptyclass[x=0, y=6]{SemanticNodeVisitor}
	\end{umlpackage}
	\begin{umlpackage}[x=2.5,y=0]{codegen}
		\umlemptyclass[x=0, y=6]{CodegenVisitor}
	\end{umlpackage}
	
	\begin{umlpackage}[x=0,y=0]{symbols}    
	    \umlclass[x=0pt,y=0pt]{Symbols}{
			- global: GlobalScope\\
			- scopes: stack$\langle$Scope$\rangle$
		}{
		    + defineArray(name: string, type: Type): void \\
            + defineConstant(name: string, t: Type, val: int): void \\
		    + defineNamedValue(name: string, t: Type): void \\
		    \\
		    + resolve(name: string): Symbol \\
            + resolveClass(name: string): Class \\
            + resolveMethod(name: string): Method \\
            + resolveNamedValue(name: string): NamedValue \\
            + resolveType(name: string): Type \\  
            \\
            + enterProgramScope(name: string): Program \\
            + enterClassScope(name: string):  Class \\
            + enterMethodArgumentsScope(): MethodArguments \\
            + enterMethodScope(name: string, returnType: Type, arguments: MethodArguments): Method \\
            
            + leaveScope(): void
		}
		\umlemptyclass[x=-2, y=-5]{Symbol}
	    \umlemptyclass[x=2, y=-5]{Scope}
	    
	    \umlVHVuniassoc[anchor1=center, anchor2=north]{Symbols}{Symbol}
	    \umlVHVuniassoc[anchor1=center, anchor2=north]{Symbols}{Scope}
	\end{umlpackage}
	\umlVHVuniassoc[anchor1=south, anchor2=north, weight=0.2]{SemanticNodeVisitor}{Symbols}
	\umlVHVuniassoc[anchor1=south, anchor2=north, weight=0.2]{CodegenVisitor}{Symbols}
\end{tikzpicture}
	\caption{'Symbols' klasa je fasada ka simbolima i opsezima vidljivosti}
	\label{fig:symbols-facade}
\end{figure}


\section{Generisanje koda}

LLVM program se sastoji od modula.
Svaki moul predstavlja jednu kompilacionu jedinicu izvornog jezika i sastoji se od funkcija, globalnih promenljivih i deklaracija tipova i spoljnih imena.

U slučaju MicroJave, ceo program sa svim predefinisanim funkcijama smešten je u jedan LLVM modul.

Dok se po drugi put prolazi kroz sintaksno stablo, pozivaju se metode LLVM API-ja kojima se postepeno generiše model programa predstavljen u memoriji.
Primeri generisanog koda su reprezentacija memorijskog modela u LLVM sintaksi.

Deklarišu se sledeće funkcije iz standardne C biblioteke jer se implementacija oslanja na njih:

\begin{lstlisting}
declare i8* @malloc(i64)
declare i32 @putchar(i32)
declare i32 @__isoc99_scanf(i8*, ...)
declare i32 @printf(i8*, ...)
declare i32 @getchar()
\end{lstlisting}

Definisane su i sledeće globalne konstante:

\begin{lstlisting}
; end of line character
@eol = constant i8 10
; null pointer
@null = constant i8* zeroinitializer
; "%d" string used as an argument for scanf/printf when reading and writing integers
@mj.dec_fmt = private constant [3 x i8] c"%d\00"
\end{lstlisting}

Predefinisane funkcije jezika su takođe umetnute direktno u modul.

Nizovi su implementirani kao struktura \mj{\{ i32, i8* \}} za dužinu niza i pokazivača na same elemente, pa je \mj{len()} funkcija implementirana kao prosto čitanje polja iz strukture.

\begin{lstlisting}
define i32 @len({ i32, i8* }) {
entry:
  ; getting pointer of first field of a struct
  %len_ptr = getelementptr { i32, i8* }* %0, i32 0, i32 0
  ; loading the value and returning it
  %len = load i32* %len_ptr
  ret i32 %len
}
\end{lstlisting}

Funkcije \mj{ord()} i \mj{chr()} su implementirane kao proširenje celobrojnog tipa iz 8 u 32 bita, odnosno zadržavanje samo nižih 8 bita 32-bitnog broja:

\begin{lstlisting}
define i32 @ord(i8) {
entry:
  %1 = sext i8 %0 to i32
  ret i32 %1
}

define i8 @chr(i32) {
entry:
  %1 = trunc i32 %0 to i8
  ret i8 %1
}
\end{lstlisting}

Koristeći popunjenu tabelu simbola, prolazi se još jednom kroz sintaksno stablo i generišu LLVM instrukcije.

Funkcije, tipovi, globalne promenljive i konstante unutar MicroJava programa su takođe definisani na globalnom nivou unutar modula.
Da bi se izbegla kolizija sa već predefinisanim imenima, na početak imena ovih entiteta se dodaje prefiks.
U zavisnosti od trenutnog opsega vidljivosti, ime će se razrešiti u odgovarajuću LLVM vrednost.

Klase iz programa se definišu u vidu LLVM struktura.
Na primer klasa \mj{List} u programu \mj{Example} koja se sastoji od jednog celobrojnog polja i reference na naredni element liste bi bila definisana na sledeći način:

\begin{lstlisting}
%"Example::List" = type { %"Example::List"*, i32 }
\end{lstlisting}

Primer definisane celobrojne konstante i promenljive korisnički definisanog tipa:

\begin{lstlisting}
@"Example::MAX_SIZE" = constant i32 100
@"Example::list" = global %"Example::List"* null
\end{lstlisting}

Zatim se prelazi na generisanje tela funkcija.
Funkcija se sastoji od liste blokova, a blokovi se sastoje od liste instrukcija.
Blokovi takođe poseduju labelu koja može da se koristi kao argument instrukciji skoka.
Poslednja instrukcija svakog bloka mora biti "terminalna" odnosno instrukcija skoka ili povratak iz funkcije.

\begin{lstlisting}

define i32 @"Example::f(i32) {
entry:
  ; ...
  ; generated instructions
  ; ...
  ret i32 %result
}
\end{lstlisting}

Ključne reči \mj{read} i \mj{write} se u zavisnosti od tipa promenljive pretvori ili u poziv C funkciji \code{getchar} odnosno \code{putchar} ili, u slucaju celobrojnog tipa, \code{scanf} odnosno \code{printf}.
Primer čitanja vrednosti celobrojnog tipa:

\begin{lstlisting}
%charCount = call i32 (i8*, ...)* @__isoc99_scanf(
    i8* getelementptr inbounds ([3 x i8]* @mj.dec_fmt, i32 0, i32 0),
    i32* %iPtr
)
\end{lstlisting}

Ključna reč \mj{new} se oslanja na C funkciju \code{malloc}:

\begin{lstlisting}
; newList = new List
%voidptr.1 = call i8* @malloc(i64 16)
%newList.ptr = bitcast i8* %voidptr.1 to %"Example::List"*
store %"Example::List"* %newList.ptr, %"Example::List"** %newList
\end{lstlisting}

Aritmetičko-logičke instrukcije imaju direktne ekvivalente unutar LLVM-a.
Promenljive su predstavljene pokazivačem na vrednost koju je potrebno učitati u registar instrukcijom \code{load} pre upotrebe u izrazu.
Vrednost rezultata se dodeljuje promenljivoj instrukcijom \code{store}:

\begin{lstlisting}
; a = a + 1
%a.1 = load i32 %a
%add_tmp = add i32 %a.1, 1
store i32 %tmp, i32* %a
\end{lstlisting}

Razlog za ovakav pristup je SSA forma LLVM koda, koja ne dozvoljava da se vrednost registra menja.
Na ovaj način vrednost registra \code{\%a} je konstantna adresa, dok se na mestima gde se vrednost te adrese koristi, koriste novi privremeni registri.
Prilikom generisanja izvršnog koda, moguće je ukloniti suvišne instrukcije, kao što su uzastopni \code{store}/\code{load} parovi, propuštanjem ovako naivno izgenerisanog koda kroz optimizacione prolaze.

Instrukcije kontrole toka: \mj{if} i \mj{while} su implementirane kao niz blokova instrukcija.
U prvom bloku se nalazi kod za izračunavanje uslova i uslovnog skoka na odgovarajuću labelu.
Potom sledi blok ili lista blokova koja odgovara instrukcijama generisanim iz iskaza unutar \mj{if}/\mj{else}/\mj{while}.
Na kraju se napravi prazan blok koji se setuje kao odredište za instrukcije koje dolaze nakon instrukcija kontrola toka i na čiju labelu se skače kada želimo da izađemo iz \mj{if}/\mj{else}/\mj{while} bloka.

\begin{lstlisting}
; while (i > 0) {
;  i = i - 1;
; }
while.1:
  %i.1 = load i32 %i
  %lt_tmp = icmp sgt i32 %i.1, 0
  %br i1 %lt_tmp, label loop.1, label whilecont.1
loop.1:
  %i.2 = load i32 %i
  %sub_tmp = sub i32 %i.2, 1
  %store i32 %i.2, i32* %i
  %br label while.1
whilecont.1:
; ...
\end{lstlisting}

Pristup elementima niza i poljima klasa implementira se pomoću LLVM instrukcije za pokazivačku aritmetiku \code{getelementptr} i \code{load} odnosno \code{store} instrukcije, koja zapravo pristupi izračunatoj adresi.
Polja unutar strukture su anonimna, pa je potrebno pamtiti njihove redne brojeve koji se prosleđuju instrukciji \code{getelementptr}.
Pošto je vrednost promenljive tipa klase implementirana kao pokazvač na strukturu, prvi indeks koji se prosleđuje \code{getelementptr} instrukciji je 0, da bi se dereferencirao sam pokazivač, pa tek onda sledi redni broj polja strukture koje se učitava:

\begin{lstlisting}
%valPtr = getelementptr %"Example::List"* %listVal, i32 0, i32 1
%val = load i32* %valPtr
\end{lstlisting}

Na kraju se dodaje i metoda \code{main} koja predstavlja polaznu tačku programa.
Ona samo poziva korisnički definisanu main funkciju i vraća vrednost 0 kao znak uspešno završenog programa.

\begin{lstlisting}
define i32 @main() {
entry:
  call void @"Example::main"()
  ret i32 0
}
\end{lstlisting}

Kada se kompletira proces generisanja modula, moguće ga je dalje menjati i primenjivati optimizacione algoritme.
Rezultujući kod se može serijalizovati na disk za kasnije izvršavanje ili izvršiti odmah prosleđivanjem ugrađenom JIT kompajleru.
Ova implementacija MicroJave se sastoji od dva odvojena izvršna programa: \code{mjc} i \code{mji}.
Program \code{mjc} je kompajler i u zavisnosti od opcija će generisati LLVM tekstualnu reprezentaciju, bajt-kod ili objektni fajl sa mašinskim instrukcijama.
Program \code{mji} je interpreter i odmah po generisanju memorijskog modela prelazi na izvršavanje.

\section{Ograničenja i specifičnosti implementacije}

Ovaj rad implementira uprošćenu verziju MicroJave koja ne podržava neke korisne konstrukte kao što su: 
\begin{itemize}
\item objektno orijentisano programiranje
\item dodelu \mj{boolean} vrednosti promenljivima
\item \mj{string} tip i literale
\end{itemize}

Svaki program sadrzi celu kopiju standardne biblioteke, umesto da je biblioteka zaseban modul koji se povezuje sa glavnim programom.

Komanda \mj{mjc} ume samo da generiše objektni fajl, dok se povezivanje u izvršni fajl mora zasebno izvršiti pomoću zasebnog programa.

Još jedna specifičnost implementacije je da specifikacija stavlja ograničenja na veličinu ulaznog fajla, broj promenljivih i slično.
Nikakve provere tog tipa nisu implementirane u okviru ovog rada, tako da su ograničenja programa, ograničenja računara na kome se prevodilac izvršava.
