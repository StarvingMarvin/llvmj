\chapter{Implementacija MicroJava prevodioca}
\label{ch:implementacija}

Ovo poglavlje opisuje arhitekturu i implementacione detalje. Posebna pažnja je posvećena implementaciji semantike jezika.

\section{Parser}

Kako ANTLR generiše C kod, rezultat parsiranja programa je struktura koja predstavlja sintaksno stablo.
Za udobniji rad  u C++u sa dobijenom strukturom implementirane su
\mj{AstWalker}, \mj{NodeVisitor} i \mj{nodeiterator} klase i nekoliko pomoćnih funkcija. \mj{AstWalker} prihvata registraciju podrazumevanog posetioca kao i instancu posetioca po tipu AST noda.
Klasa \mj{nodeiterator} služi za obilazak dece datog noda.

\section{Semantika}

\subsection*{Simboli}

Centralno mesto u ovoj fazi MicroJava prevodioca zauzima hijerarhija klasa vezanih za simbole, opsege i tabelu simbola.

U MicroJavi možemo izdvojiti nekoliko vrsta simbola, kao što su primitivni tipovi, promenljive, konstante, klase i metode.
Radi što lakše analize, a kasnije i generisanja koda, ovi simboli predstavljeni su hijerarhijom klasa kao na dijagramu \ref{fig:symbols}.

\begin{figure}[h]

	\centering
	\begin{tikzpicture}
	\tikzumlset{font=\footnotesize}

	\begin{umlpackage}[x=0,y=0]{semantics}
		\umlclass[x=210pt,y=250pt]{Symbol}{+ name : string }{} 
		\umlclass[x=90pt,y=165pt]{Type}
			{}{+ operator==(t: Type) : bool \\ + compatible(t: Type) : bool}
		\umlclass[x=290pt,y=176pt]
			{NamedValue}{}{}
		\umlclass[x=210pt,y=176pt]
			{Program}{}{}
		\umlclass[x=40pt,y=85pt]
			{ReferenceType}{}{}
		\umlclass[x=10pt,y=16pt]
			{Array}{}{}
		\umlclass[x=70pt,y=16pt]
			{Class}{}{}
		\umlclass[x=140pt,y=85pt]
			{MethodType}{}{}
		\umlclass[x=250pt,y=85pt]
			{Method}{}{}
		\umlclass[x=330pt,y=80pt]
			{Constant}{+ value : int}{}
		\umlVHVinherit[anchor1=north, anchor2=south, arm2=-15pt]
			{Type}{Symbol}
		\umlVHVinherit[anchor1=north, anchor2=south, arm2=-15pt]
			{NamedValue}{Symbol}
		\umlVHVinherit[anchor1=north, anchor2=south, arm2=-15pt]
			{Program}{Symbol}
		\umlVHVinherit[anchor1=north, anchor2=south, arm2=-15pt]
			{ReferenceType}{Type}
		\umlVHVinherit[anchor1=north, anchor2=south, arm2=-15pt]
			{MethodType}{Type}
		\umlVHVinherit[anchor1=north, anchor2=south, arm2=-15pt]
			{Array}{ReferenceType}
		\umlVHVinherit[anchor1=north, anchor2=south, arm2=-15pt]
			{Class}{ReferenceType}
		\umlVHVinherit[anchor1=north, anchor2=south, arm2=-30pt]
			{Constant}{NamedValue}
		\umlVHVinherit[anchor1=north, anchor2=south, arm2=-30pt]
			{Method}{NamedValue}
		\umlVHVuniaggreg[attr=methodType|1, anchor1=-120, anchor2=-60,arm2=-20pt,pos=1.9,align=left]
			{Method}{MethodType}
		\umlVHuniaggreg[attr=type|1,anchor1=-130, anchor2=-20,pos=1.9,align=left]
			{NamedValue}{Type}
	
	\end{umlpackage}
\end{tikzpicture} 
	\caption{Pojednostavljena struktura klasa koje predstavljaju simbole}
	\label{fig:symbols}
\end{figure}

Prilikom analize programa, bitno je da za tipove vrednosti u nekom izrazu utvrdimo da li su jednaki očekivanom tipu, odnosno da li su kompatabilni u slučaju dodele.
Klasa \code{Type} deklariše virtualne metode \code{operator==} i \code{compatible}, a podrazumevana implementacija odgovara semantici primitivnih tipova: tipovi su jednaki ako se isto zovu, a kompatibilni su samo ako su jednaki.

Klasa \code{ReferenceType} prihvata \mj{null} kao compatibilan tip.

Tipovi nizova i metoda se implicitno registruju u globalnu tabelu simbola prilikom prvog nailaženja u kodu.

MicroJava metod je implementiran kao vrednost koja ima tip.
Tip metoda sastoji se od povratnog tipa i tipa argumenata. 
U fazi semantičke analize, tip se koristi za proveru argumenata i dodele rezultata promenjvoj prilikom poziva metode.
Kasnije se u fazi generisanja koda ponovo koristi za pravljenje ekvivalentnog LLVM tipa.
Metode takodje poseduju i dva opsega vidljivosti: za lokalne promenjive i argumente.

\begin{figure}[h]
	\centering
	\begin{tikzpicture}
	\tikzumlset{font=\footnotesize}

	\begin{umlpackage}[x=0,y=0]{symbols}
	
		\umlclass[x=0pt,y=0pt]{Scope}{
			}{
				+ define(s: Symbol): void \\ 
				+ resolve(name: string): Symbol \\
				+ begin(): iterator \\
				+ end(): iterator
			}
		
		\umlemptyclass[x=-140pt,y=22pt]{Symbol}
		
		\umlclass[x=-86pt,y=-115pt]{MethodArguments}
			{}{
				+ matchArguments(args: vector)
			}

		\umlclass[x=180pt,y=-23pt]{SplitScope}
			{}{
				+ methodBegin(): iterator \\
				+ methodEnd(): iterator \\
				+ variableBegin(): iterator \\
				+ variableEnd(): iterator \\
				+ constantBegin(): iterator \\
				+ constantEnd(): iterator \\
				+ classBegin(): iterator \\
				+ classEnd(): iterator
			}
		
%		\umlemptyclass[x=-160pt,y=-50pt]{SplitScope}
		
		\umlemptyclass[x=60pt,y=-110pt]{ClassScope}
		
		\umlemptyclass[x=-140pt,y=-40pt]{Method}
		
		\umlemptyclass[x=80pt,y=-160pt]{Class}
		
		\umlemptyclass[x=120pt,y=-160pt]{Program}
		
		\umlemptyclass[x=210pt,y=-160pt]{GlobalScope}
		
	\end{umlpackage}
\end{tikzpicture}
	\caption{Pojednostavljena struktura klasa koje implementiraju i koriste opsege vidljivosti}
	\label{fig:scopes} 
\end{figure}

Opsezi vidljivosti takodje imaju svoju hijerarhiju, prikazanu na slici \ref{fig:scopes}.
Osnovna funkcionalnost: definisanje i pronalaženje simbola, implementirana je u klasi \mj{Scope}.
Globalni opseg, implementiran u klasi \mj{GlobalScope}, sadrži sve predefinisane tipove i simbole, kao i implicitno generisane tipove nizova i potpise metoda.
Globalni opseg takođe sadrži i referencu na objekat tipa \mj{Program} koja predstavlja spoljnu klasu MicroJava programa.
Ulazni fajl je nevalidan ako do kraja njegove obrade nije definisana tačno jedna instanca klase \mj{Program}.


\subsection*{Semantička analiza}

Koristeći \mj{AstWalker} klasu definisanu u \mj{parser} paketu, za svaki tip tokena iz sintaksnog stabla definisana je \mj{Visitor} klasa koja proverava semantičku ispravnost izraza i sakuplja simbole. 
Svi posetioci poseduju referencu na objekat tipa \mj{Symbols} koji predstavlja fasadu ka konkretnim simbolima i opsezima vidljivosti \ref{fig:symbols-facade}.
Sa uspešno izvršenom proverom ulaznog programa i popunjenom tabelom simbola, sve je spremno za generisanje koda.
\begin{figure}[h]
	\centering
	\begin{tikzpicture}
	\tikzumlset{font=\footnotesize}
	\begin{umlpackage}[x=0,y=0]{semantics}
		\umlclass[x=0pt,y=0pt]{Symbols}{}{}
	\end{umlpackage}
\end{tikzpicture}
	\caption{'Symbols' klasa je fasada ka simbolima i opsezima vidljivosti}
	\label{fig:symbols-facade}
\end{figure}


\section{Generisanje koda}

Ceo MicroJava program sa svim predefinisanim funkcijama smešten je u jedan LLVM modul. Implementacija se oslanja na sledeće funcije iz standardne C biblioteke: \mj{i8* @malloc(i64)},
\mj{i32 @putchar(i32)},
\mj{i32 @\_\_isoc99\_scanf(i8*, ...)},
\mj{i32 @printf(i8*, ...)} i
\mj{i32 @getchar()}.

Nizovi su implementirani kao struktura \mj{\{ i32, i8* \}} za dužinu niza i same elemente, pa je \mj{len()} funkcija implementirana kao prosto čitanje polja iz strukture.

Slično tome, \mj{ord()} i \mj{chr()} funkcije su implementirane kao proširenje celobrojnog tipa iz 8 u 32 bita, odnosno zadržavanje samo nižih 8 bita 32-bitnog broja.

Koristeći popunjenu tabelu simbola, prolazi se još jednom kroz sintaksno stablo i generišu LLVM instrukcije.
