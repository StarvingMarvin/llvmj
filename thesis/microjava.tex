
\chapter{MicroJava}

MicroJava je jezik koji ću implementirati u okviru ovog rada. Sintaksom podseća na Javu, ali je mnogo jednostavnijii i semantički je bliži proceduralnom jeziku kao sto je Pascal. Sledi kratak pregled ovog jezika kako je definisan u materijalima kursa programskih prevodioca.

Jezik se sastoji od sledećih tipova simbola: promenljive, tipovi i metode. MikroJava program se sastoji od jedne klase koja ima statička polja i statičke metode. Ne postoje druge spoljašnje klase, već samo unutrašnje koje se mogu koristiti kao tipovi podataka. Validan  MikroJava program mora da ima definisanu metodu \mj{void main()} koja se poziva prilikom pokretanja programa.

\subsection*{Iskazi}

Iskaz\footnote{\eng Statement} je najmanja samostalna jedinica nekog imperativnog programskog jezika. Metode MicroJave se sastoje od niza iskaza. Iskaz može biti poziv neke metode, dodela vrednosti nekog izraza\footnote{\eng Expression} memorijskoj odrednici\footnote{\eng Designator} ili neki od iskaza za kontrolu toka. Memorijske odrednice se nazivaju i L-vrednosti jer se mogu naći sa leve strane znaka jednakosti. U MicroJavi odrednice mogu biti promenjive, elementi niza ili polja klase.

\subsection*{Tipovi}

Od prostih tipova MicroJava podržava 32-bitne i 8-bitne celobrojne vrednosti (\mj{int} i \mj{char}), to su ujedno i jedini tipovi literala koje jezik prepoznaje. Od agregatnih tipova postoje jednodimenzionalni nizovi i korisnički definisane  klase koje po svojoj semantici podsećaju na strukture u C-u. Logički tip (\textit{boolean}) postoji samo kao argument kontrolnim strukturama, što su u MicroJavi \mj{if} i \mj{while}. Slično Javi, nizovi i klase su tipa reference.

Tipovi su ekvivalentni ako imaju isto ime ili ako su oba tipa nizovi, a tipovi njihovih elemenata su ekvivalentni. Tipovi su kompatabilni ako ekvivalentni ili ako je jedan od njih tipa reference a drugi tipa \mj{null}. Logički operatori mogu biti primenjeni samo ako su izrazi sa obe strane operatora kompatibilni. Dodela je moguća ako su tipovi na levoj i desnoj strani ekvivalentni ili ako je tip na levoj strani tipa reference, a na desnoj tipa \mj{null}.

\subsection*{Ključne reči}

Jedanaest ključnih reči, koliko MicroJava poseduje može se svrstati u nekoliko grupa:

\begin{description}
    \item[Kontrola toka] \hfill \\
    Uslovno izvršavanje je moguće pomoću \mj{if/else} iskaza. Petlje se opisuju sa \mj{while}, a moguć je prevremeni izlazak iz petlje sa \mj{break}. Povratak iz metode se vrši sa \mj{return}.

   \item[Ulaz i izlaz] \hfill \\
	kako nije moguće definisati metode promenljivog broja parametara, niti je moguće prosleđivanje prostih tipova po referenci, \mj{print} i \mj{read} su definisani kao ključne reči u samom jeziku.

	\item[Tipovi] \hfill \\
	Konstante se deklarišu modifikatorom \mj{final} koji je moguće primeniti na tipove int i char. Metode bez povratne vrednosti imaju tip \mj{void}. Korisnički tipovi se deklarišu sa \mj{class}, a zajedno sa nizovima instanciraju sa \mj{new}.

\end{description}


\subsection*{Operatori}



\subsection*{Predeklarisana imena}

Osim ključnih reči, MicroJava poseduje i nekoliko predeklarisanih imena, od kojih su neka već pomenuta: tipovi \mj{int} i \mj{char}, kao i konstanta \mj{null}. Deklarisani su još i konstanta za kraj reda \mj{eol} koja odgovara znaku '\textbackslash n', kao i tri metode: \mj{chr()}, \mj{ord()} i \mj{len()}. Metoda  \mj{chr()} uzima celobrojnu vrednost i vraća odgovarajući ASCII karakter. Nasuprot tome \mj{ord()} prima karakter kao parametar i vraća odgovarajući celobrojnu reprezentaciju. Metoda \mj{len()} uzima niz i vraća njegovu dužinu. 

\subsection*{Opseg važenja}

U MicroJavi postoje tri leksički ugnježdena opsega: globalni opseg (\mj{universe}) koji sadrži predeklarisana imena i glavnu klasu. Opseg glavne klase u kojme se definisu korisnički tipovi, promenljive i metode i opsezi samih metoda. Kako svako ime mora biti deklarisano pre prve upotrebe, indirektna rekurzija, kako u pozivima metoda, tako i u definisanju klasa, nije moguća. Ponovno deklarisanje imena u unutrašnjem opsegu $S$, sakriva deklaraciju tog imena u spoljnjem opsegu. Imena simbola moraju biti deklarisana pre prvog korišćenja, a jedno ime ne sme biti deklarisano više puta u okviru istog opsega.
