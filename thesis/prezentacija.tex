\documentclass{beamer}

\usepackage[serbian]{babel}
\usepackage[utf8x]{inputenc}
\usepackage{amsmath}
\usepackage{algorithm}
\usepackage{algpseudocode}
\usepackage{graphicx}
\usepackage{listings}
\usepackage{tikz}
\usepackage{tikz-uml}
\usepackage{tkz-graph}
\usepackage[font={small,it}]{caption}
\usetikzlibrary{arrows,decorations.pathmorphing,backgrounds,positioning,fit,petri, automata}

\newcommand{\LL}[1]{$LL(#1)$ {}}
\newcommand{\LLk}{\LL{k}}
\newcommand{\LLa}{\LL{\ast}}

\newcommand{\code}[1]{\texttt{#1}}

\usetheme{PaloAlto}
\usecolortheme{seahorse}
\usefonttheme{structurebold}

\lstset{
  basicstyle=\footnotesize\ttfamily
}

\setbeamertemplate{footline}
{
}
%\setbeameroption{show notes on second screen}

\title{MicroJava na LLVM infrastrukturi}
\author{Luka Stojanović}
\institute{
Elektrotehnički fakultet, \\ Beogradski univerzitet
}
\date{12. 07. 2016.}
\subject{Software engineering}

\begin{document}
\begin{frame}
	\titlepage
\end{frame}
  
\begin{frame}
	\frametitle{Pregled}
    \tableofcontents
\end{frame}

\section{Uvod}

\begin{frame}
	\frametitle{Uvod}
	\begin{itemize}
		\item Prepoznavanje ulaznog teksta
	    \begin{itemize}
			\item Prvi prevodioci nastali pre skoro 60 godina
			\item $LR$ algoritam opisan pre nešto više od 50 godina
		\end{itemize}
		\item Implementacija odlika jezika
		\item Generisanje izlaznog koda
	    \begin{itemize}
			\item Svaki prevodilac implementirao a sebe
			\item Virtualne mašine
			\item LLVM
		\end{itemize}
	\end{itemize}
\end{frame}


\begin{frame}
	\frametitle{MicroJava}
\end{frame}


\section{Korišćeni alati}
% antlr
\begin{frame}
	\frametitle{ANTLR}
	\begin{itemize}
		\item \LLa parser
		\item Implementiran u Javi
		\item Ume da generiše kod za više jezika (Java, C, C\#, itd.)
		\item Ista gramatika za lekser i parser
		\item Razvojno okruženje
		\item Generisanje sintaksnog stabla
		\item Prevođenje iz teksta u tekst
	\end{itemize}
\end{frame}

\begin{frame}[fragile]
	\frametitle{ANTLR - Sintaksno stablo}
	
	\begin{lstlisting}
program 
: CLASS IDENT (const_decl|var_decl|class_decl)*
  '{' method_decl* '}'
  -> ^(PROGRAM IDENT
       const_decl* class_decl*
       var_decl* method_decl*);

statement
    :   while_stat
        | BREAK ';'!
        | RETURN^ expr? ';'!
        | READ^ '('! designator ')'! ';'!
        | print_stat
        | des_stat
        | '{' statement* '}' -> ^(BLOCK statement*)
        | if_stat;
	\end{lstlisting}
	

\end{frame}


% llvm
\begin{frame}
	\frametitle{LLVM}
	\begin{itemize}
		\item Pravljenje kompajlera i alata oko kompajlera
		\item Clang, Rust, Rubinius, $\ldots$
		\item lldb, lld, $\ldots$
		\item klee, vmkit, gragonegg, $\ldots$
		\item API, "asemblerski" kod, binarni kod
		\item Generisanje mašinskog koda za više različitih platformi
	\end{itemize}
\end{frame}

\section{Detalji implementacije}
% parsiranje
\begin{frame}
	\frametitle{Parsiranje}
	\begin{itemize}
		\item ANTLR generiše sintaksno stablo kao strukturu u memoriji.
		\item Svaki čvor ima tip
		\item Detektovanje greške ne zaustavlja rad parsera
		\item \code{AstWalker}-u se registruje posetilac za svaki tip
	\end{itemize}
\end{frame}

% parsiranje - slika

\begin{frame}[c]
	\frametitle{Parsiranje}
\begin{center}

	\scalebox{0.6}{
		\begin{tikzpicture}
	\tikzumlset{font=\footnotesize}

	\begin{umlpackage}[x=0,y=0]{parser}

		\umlclass[x=20pt,y=20pt]{AstWalker}
			{
				- stack: vector$\langle$AST$\rangle$		
			}{
	            + addVisitor(tokenType: uint32\_t, visitor: NodeVisitor): void \\
	            + getVisitor(uint32\_t tokenType): NodeVisitor \\
	            + visit(AST ast): void \\
	            + tokenType(): uint32\_t \\
	            + tokenText(): char$\ast$ \\
	            + childCount(): size\_t \\
	            + nilNode(): bool \\
	            + printPosition(ostream\& os): ostream\& \\
	            + firstChild(): nodeiterator \\
	            + lastChild(): nodeiterator \\
	            + currentNode(): AST \\
	            + template $\langle$class T$\rangle$ setData(T$\ast$ data): void \\
	            + template $\langle$class T$\rangle$ getData(): T$\ast$
			}

		\umlclass[x=20pt,y=200pt]{NodeVisitor}
			{}{
				+ operator()(walker: AstWalker): void
			}
			
		\umlemptyclass[x=250pt,y=20pt]{nodeiterator}
		
		\umlclass[x=250pt,y=190pt]{std::iterator}
			{}{
				+ operator++() \\
			    + operator*() \\
			    ...
			}

		
		\umluniaggreg[attr=defaultVisitor|1, anchor1=54,anchor2=-20]
			{AstWalker}{NodeVisitor}

    	\umluniaggreg[attr=visitors|*,pos=0.5]
			{AstWalker}{NodeVisitor}

		\umldep[anchor1=-160,anchor2=126]
			{NodeVisitor}{AstWalker}
			
		\umlimpl{nodeiterator}{std::iterator}
		
		\umluniassoc{AstWalker}{nodeiterator}		
		
%		\umlVHuniaggreg[attr=type|1,anchor1=-130, anchor2=-16,pos=1.9,align=left]
%			{NamedValue}{Type}
%	
%		\umlHVHuniaggreg[attr=returnType|1, anchor1=10,  anchor2=-16, arm2=40pt, pos=1.4]
%			{MethodType}{Type}
%		\umlHVHuniaggreg[attr=*|arguments, anchor1=-10,  anchor2=-16, arm2=55pt, pos=1.8]
%			{MethodType}{Type}
	\end{umlpackage}
\end{tikzpicture} 
	}
\end{center}
\end{frame}


% semantika

\begin{frame}
	\frametitle{Semantička analiza}
	\begin{itemize}
		\item Obilazak stabla
		\item Skupljanje simbola
		\item Skupljanje implicitnih tipova potrebnih za generisanje koda
		\begin{itemize}
			\item Tipovi niza
			\item Tipovi metoda
		\end{itemize}
		\item Provera pravila jezika:
		\begin{itemize}
		    \item Sva imena su deklarisana pre prve upotrebe
			\item Tipovi u izrazima su kompatibilni
			\item Dodela samo promenljivima
			\item \code(break) samo unutar petlje
			\item \code{return} postoji u funcijama koje proizvode vrednost
			\item \code(main) postoji, ne prima argumente i ne proizvodi vrednost
		\end{itemize}
	\end{itemize}
\end{frame}


% semantika - slika

\begin{frame}[c]
	\frametitle{Simboli}
	\begin{center}
	\scalebox{0.6}{
		\begin{tikzpicture}
	\tikzumlset{font=\footnotesize}

	\begin{umlpackage}[x=0,y=0]{symbols}

		\umlclass[x=210pt,y=250pt]{Symbol}
			{+ name : string }{} 

		\umlclass[x=90pt,y=165pt]{Type}
			{}{
				+ operator==(t: Type) : bool \\ 
				+ compatible(t: Type) : bool
			}

		\umlemptyclass[x=290pt,y=176pt]{NamedValue}

		\umlemptyclass[x=210pt,y=176pt]{Program}

		\umlemptyclass[x=40pt,y=85pt]{ReferenceType}

		\umlemptyclass[x=10pt,y=20pt]{Array}

		\umlemptyclass[x=70pt,y=20pt]{Class}

		\umlemptyclass[x=140pt,y=85pt]{MethodType}

		\umlemptyclass[x=250pt,y=85pt]{Method}

		\umlclass[x=330pt,y=80pt]{Constant}
			{+ value : int}{}

		\umlVHVinherit[anchor1=north, anchor2=south, arm2=-15pt]
			{Type}{Symbol}

		\umlVHVinherit[anchor1=north, anchor2=south, arm2=-15pt]
			{NamedValue}{Symbol}

		\umlVHVinherit[anchor1=north, anchor2=south, arm2=-15pt]
			{Program}{Symbol}

		\umlVHVinherit[anchor1=north, anchor2=south, arm2=-15pt]
			{ReferenceType}{Type}

		\umlVHVinherit[anchor1=north, anchor2=south, arm2=-15pt]
			{MethodType}{Type}

		\umlVHVinherit[anchor1=north, anchor2=south, arm2=-15pt]
			{Array}{ReferenceType}

		\umlVHVinherit[anchor1=north, anchor2=south, arm2=-15pt]
			{Class}{ReferenceType}

		\umlVHVinherit[anchor1=north, anchor2=south, arm2=-40pt]
			{Constant}{NamedValue}

		\umlVHVinherit[anchor1=north, anchor2=south, arm2=-40pt]
			{Method}{NamedValue}

		\umlVHVuniaggreg[attr=methodType|1, anchor1=-120, anchor2=-60,arm2=-20pt,pos=1.9,align=left]
			{Method}{MethodType}

		\umlVHuniaggreg[attr=type|1,anchor1=-130, anchor2=-16,pos=1.9,align=left]
			{NamedValue}{Type}
	
		\umlHVHuniaggreg[attr=returnType|1, anchor1=10,  anchor2=-16, arm2=40pt, pos=1.4]
			{MethodType}{Type}
		\umlHVHuniaggreg[attr=*|arguments, anchor1=-10,  anchor2=-16, arm2=55pt, pos=1.8]
			{MethodType}{Type}
	\end{umlpackage}
\end{tikzpicture} 
	}
	\end{center}
\end{frame}


% generisanje koda

\begin{frame}
	\frametitle{Generisanje koda}
	\begin{itemize}
	    \item Ceo program sa standardnom bibliotekom u jednom modulu
		\item Registruju se tipovi iz tabele simbola, globalne konstante i promenljive
		\item Prolazak kroz tela funkcija
		\item Provera izpravnosti modula i izvršavanje optimizacija
	\end{itemize}
\end{frame}

% primeri

\begin{frame}[fragile]
	\frametitle{Generisanje koda - standardna biblioteka}
	\begin{lstlisting}
define i32 @len({ i32, i8* }) {
entry:
  ; getting pointer of first field of a struct
  %len_ptr = getelementptr { i32, i8* }*
      %0, i32 0, i32 0
  ; loading the value and returning it
  %len = load i32* %len_ptr
  ret i32 %len
}
\end{lstlisting}

\end{frame}

\begin{frame}[fragile]
	\frametitle{Generisanje koda - standardna biblioteka}
	\begin{lstlisting}
; while (i > 0) {
;  i = i - 1;
; }
while.1:
  %i.1 = load i32 %i
  %lt_tmp = icmp sgt i32 %i.1, 0
  %br i1 %lt_tmp, label loop.1, label whilecont.1
loop.1:
  %i.2 = load i32 %i
  %sub_tmp = sub i32 %i.2, 1
  %store i32 %i.2, i32* %i
  %br label while.1
whilecont.1:
; ...
\end{lstlisting}
\end{frame}


\section{Dalja unapređenja}
% zakljucak
\begin{frame}
	\frametitle{Dalja unapređenja}
	\begin{itemize}
	\item Objektno orijentisano programiranje
	\item Tipovi \code{boolean} i \code{string}
	\item Linker
	\item Biblioteka kao zaseban modul
	\end{itemize}
\end{frame}

\end{document}
