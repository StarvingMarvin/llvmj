\chapter{Generisanje koda}

\section{Uvod u LLVM}

%Because of the differing goals and representations, LLVM
%is complementary to high-level virtual machines (e.g., Small-
%Talk [18], Self [43], JVM [32], Microsoft’s CLI [33], and oth-
%ers), and not an alternative to these systems. It differs from
%these in three key ways. First, LLVM has no notion of high-
%level constructs such as classes, inheritance, or exception-
%handling semantics, even when compiling source languages
%with these features. Second, LLVM does not specify a
%runtime system or particular object model: it is low-level
%enough that the runtime system for a particular language
%can be implemented in LLVM itself. Indeed, LLVM can
%be used to implement high-level virtual machines. Third,
%LLVM does not guarantee type safety, memory safety, or
%language interoperability any more than the assembly lan-
%guage for a physical processor does.
%The LLVM compiler framework exploits the code repre-
%sentation to provide a combination of five capabilities that
%we believe are important in order to support lifelong anal-
%ysis and transformation for arbitrary programs. In general,
%these capabilities are quite difficult to obtain simultaneously,
%but the LLVM design does so inherently:
%(1) Persistent program information: The compilation model
%preserves the LLVM representation throughout an ap-
%plication’s lifetime, allowing sophisticated optimiza-
%tions to be performed at all stages, including runtime
%and idle time between runs.
%(2) Offline code generation: Despite the last point, it is
%possible to compile programs into efficient native ma-
%chine code offline, using expensive code generation
%techniques not suitable for runtime code generation.
%This is crucial for performance-critical programs.
%(3) User-based profiling and optimization: The LLVM
%framework gathers profiling information at run-time in
%the field so that it is representative of actual users, and
%can apply it for profile-guided transformations both at
%run-time and in idle time1 .
%(4) Transparent runtime model: The system does not
%specify any particular object model, exception seman-
%tics, or runtime environment, thus allowing any lan-
%guage (or combination of languages) to be compiled
%using it.
%(5) Uniform, whole-program compilation: Language-indep-
%endence makes it possible to optimize and compile all
%code comprising an application in a uniform manner
%(after linking), including language-specific runtime li-
%braries and system libraries.


LLVM je infrastrujtura za razvijanje optimizujućih prevodilaca. Dizajniran je tako da omogući analizu i transformaciju programa kroz sve faze programskog ciklusa: od prevođenja i povezivanja, do vremena samog izvršavanja programa\cite{llvm-cgo04}. 
Iako LLVM predstavlja akronim za ,,Low Level Virtual Machine''
\footnote{eng. virtualna mašina niskog nivoa}
, sam naziv možda ne predstavlja najjasniju sliku u to šta LLVM pruža, a gde njegov domen prestaje. 
Pojam ,,virtualna mašina'' se često vezuje za Java ili .net virtualnu mašinu koje definišu koncepte višeg nivoa kao što su klase, nasleđivanje, izuzeci ili automatsko upravljanje memorijom. 
Sve ove stvari su izvan opsega LLVM-a, koji sa druge strane može biti korišćen da se takve virtualne mašine implementiraju. 
Još jedna stvar koja se podrazumeva kod drugih virtualnih mašina je da je njihov bajt-kod je portabilan između platformi. 
Sa druge strane kako bi LLVM omogućio generisanje efikasnog mašinskog koda za konkretnu procesorsku arhitekturu, on mora da omogući tvorcima kompajlera pristup specifičnim operacijama podržanih arhitektura. 

Neki od projekata koji se baziraju ili koriste LLVM su Clang \ndash kompajler za C, C++ i Objective C, 
zatim VMKit projekat implementacije Java virtualne mašine, 
Glasgow Haskell Compiler od skora ima opciju da preko LLVM-a generiše kod, 
Rubinius implementacija Ruby-ja koristi LLVM za optimizacije i JIT
\footnote{\skr \eng Just in Time \ndash Prevođenje međukoda u mašinski kod neposredno pre izvršavanja} 
kompajliranje.

\begin{center}

\begin{tikzpicture}
	[inner sep=2.8mm,
	box/.style={rectangle, draw, minimum width=25mm, }]
	\node[box] (c-frontend)   {C};
	\node[box] (haskell-frontend)  [below=of c-frontend]  {Haskell};
	\node[box] (other-frontend)  [below=of haskell-frontend]  {$\cdots$};
	\node[box] (optimizers)  [right=of haskell-frontend]  {Optimizacije}
		edge [<-] (c-frontend)
		edge [<-] (haskell-frontend)
		edge [<-] (other-frontend);
	\node[box] (x86-backend) [right=of optimizers] {x86}
		edge [<-] (optimizers);
	\node[box] (arm-backend) [above=of x86-backend] {ARM}
		edge [<-] (optimizers);
	\node[box] (other-backend) [below=of x86-backend] {$\cdots$}
	edge [<-] (optimizers);
\end{tikzpicture} \\

\textit{LLVM omogućava da se delovi arhitekture kompajlera kao što su optimizacije i generisanje mašinskog koda dele između nezavisnih projekata}
\end{center}

\section{LLVM međureprezentacija}

%LLVM defines a common, low-level
%code representation in Static Single Assignment (SSA) form,
%with several novel features: a simple, language-independent
%type-system that exposes the primitives commonly used to
%implement high-level language features; an instruction for
%typed address arithmetic; and a simple mechanism that can
%be used to implement the exception handling features of
%high-level languages (and setjmp/longjmp in C) uniformly
%and efficiently. The LLVM compiler framework and code
%representation together provide a combination of key capa-
%bilities that are important for practical, lifelong analysis and
%transformation of programs.

Centralno mesto u arhitekturi LLVM-a\cite{aosa} zauzima međureprezentacija
\footnote{\eng Intermediate Representation \ndash IR} 
kojom se opisuje kod unutar sistema. 
LLVM IR je dizajniran da omogući analize i transformacije koda kakve se mogu očekivati u delu kompajlera koji se bavi optimizacijom. 
Ono što je značajno kod ove reprezentacije je da ne predstavlja neki interni implementacioni detalj kao kod vecine kompajlera, 
nego je pogurana u prvi plan kao jezik jasno definisane semantike. 
LLVM međureprezentacija se tako sasvim ravnopravno može opisati svojom sintaksom sličnom asembleru, 
bajt-kodom i direktno iz programa putem C++ API-ja
\footnote{\eng Application Programming Interface}.

Značajne osobine LLVM reprezentacije su 
SSA forma
\footnote{\skr \eng Static single asignement \ndash Statička jedinstvena dodela}
striktno tipiziranje podataka
pristup operacijama niskog nivoa, uključujući i instrukcije specifične za pojedine procesore.


\section{LLVM tipovi}

Tipovi podataka LLVM-u omogućavaju brojne optimizacije da budu spovedene direktno nad međureprezentacijom, 
bez potrebe za se vrši uporedna analiza mogućih vrednosti u nekom registru pre primene transformacije.
LLVM je striktan po pitanju tipova u smislu da ni u jednom trenutku neće izvršiti implicitnu konverziju između dva tipa, 
uključujući tu i proširenje brojnih vrednosi ili pretvaranje jednog tipa pokazivača u drugi. 
Sa druge strane korisnika ništa ne sprečava da eksplicitno izvrši konverziju između bilo koja dva tipa odgovarajućim instrukcijama.

Tipovi se mogu podeliti na primitivne kao što su tipovi celobrojnih i realnih brojeva, labela, i metapodatak; 
i izvedene tipove kao što su nizovi, funkcije, pokazivači, strukture i vektori.

Osim ove podele bitno je istaći da nisu svi tipovi ravnopravni.
Oni tipovi koji mogu biti rezultat LLVM instrukcija i mogu se skladištiti u memoriji, prosleđivati kao parametar ili vraćati kao rezultat funkcije, nazivaju se tipovi prvog reda, a to su: celobrojni i realni tipovi, pokazivači, nizovi, strukture, vektori, labele i metapodaci.


\subsection*{Primitivni tipovi}

Primitivni tipovi su osnovni gradivni blokovi LLVM sistema.

Pošto tipovi celih i realnih brojeva zavise od podrške konkretne procesorske arhitekture,
sam LLVM pruža veliku fleksibilnost u njihovom deklarisanju
pa su mogući tipovi bilo koje širine od 1 do $2^{23} - 1$ (pribliižno 8 miliona) koji se označavaju sa \code{iN} gde \code{N} predstavlja širinu u bitima. Posebna instrukcija mapira pojedine tipove na 

Postoji 5 tipova realnih brojeva: 
\code{float} koji je širine 32 bita, 
\code{double} širine 63 bita, 
\code{fp128} od 128 bita sa mantisom od 112 bita, 
kao i hardverski specifične x86_fp80 od 80 bita 
i ppc_fp128 koji takodje ima 128 bita, ali mantisu širine 64 bita.


\subsection*{Izvedeni tipovi}


% opis izvedenih tipova



\section{LLVM instrukcije}
%The LLVM instruction set captures the key operations of
%ordinary processors but avoids machine-specific constraints
%such as physical registers, pipelines, and low-level calling
%conventions. LLVM provides an infinite set of typed virtual
%registers which can hold values of primitive types (Boolean,
%integer, floating point, and pointer). The virtual registers
%are in Static Single Assignment (SSA) form [15]. LLVM
%is a load/store architecture: programs transfer values be-
%tween registers and memory solely via load and store op-
%erations using typed pointers. The LLVM memory model is
%described in Section 2.3.
%The entire LLVM instruction set consists of only 31 op-
%codes. This is possible because, first, we avoid multiple op-
%codes for the same operations3 . Second, most opcodes in
%LLVM are overloaded (for example, the add instruction can
%operate on operands of any integer or floating point operand
%type). Most instructions, including all arithmetic and logi-
%cal operations, are in three-address form: they take one or
%two operands and produce a single result.
%LLVM uses SSA form as its primary code representation,
%i.e., each virtual register is written in exactly one instruc-
%tion, and each use of a register is dominated by its definition.
%Memory locations in LLVM are not in SSA form because
%many possible locations may be modified at a single store
%through a pointer, making it difficult to construct a rea-
%sonably compact, explicit SSA code representation for such
%locations. The LLVM instruction set includes an explicit
%phi instruction, which corresponds directly to the standard
%(non-gated) φ function of SSA form. SSA form provides a
%compact def-use graph that simplifies many dataflow opti-
%mizations and enables fast, flow-insensitive algorithms to
%achieve many of the benefits of flow-sensitive algorithms
%without expensive dataflow analysis. Non-loop transforma-
%tions in SSA form are further simplified because they do
%not encounter anti- or output dependences on SSA registers.
%Non-memory transformations are also greatly simplified be-
%cause (unrelated to SSA) registers cannot have aliases.
%LLVM also makes the Control Flow Graph (CFG) of every
%function explicit in the representation. A function is a set
%of basic blocks, and each basic block is a sequence of LLVM
%instructions, ending in exactly one terminator instruction
%(branches, return, unwind, or invoke; the latter two are
%explained later below). Each terminator explicitly specifies
%its successor basic blocks.

add sub...

\section{Optimizacija koda}




\section{Elementi implementacije MicroJave}

