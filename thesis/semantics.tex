\chapter{Implementacija semantike}

Ovo poglavlje opisuje arhitekturalne odluke i klasnu hijerarhiju koja je ključna u semantičkoj analizi i implementaciji osobina jezika.
Način na koji se iz memorijskog modela generiše LLVM međureprezentacija je predmet narednog poglavlja.


\section{Simboli}

Centralno mesto u ovoj fazi MicroJava prevodioca zauzima hijerarhija klasa vezanih za simbole, opsege i tabelu simbola.

\begin{figure}[h]

	\centering
	\begin{tikzpicture}
	\tikzumlset{font=\footnotesize}

	\begin{umlpackage}[x=0,y=0]{semantics}
		\umlclass[x=210pt,y=250pt]{Symbol}{+ name : string }{} 
		\umlclass[x=90pt,y=165pt]{Type}
			{}{+ operator==(t: Type) : bool \\ + compatible(t: Type) : bool}
		\umlclass[x=290pt,y=176pt]
			{NamedValue}{}{}
		\umlclass[x=210pt,y=176pt]
			{Program}{}{}
		\umlclass[x=40pt,y=85pt]
			{ReferenceType}{}{}
		\umlclass[x=10pt,y=16pt]
			{Array}{}{}
		\umlclass[x=70pt,y=16pt]
			{Class}{}{}
		\umlclass[x=140pt,y=85pt]
			{MethodType}{}{}
		\umlclass[x=250pt,y=85pt]
			{Method}{}{}
		\umlclass[x=330pt,y=80pt]
			{Constant}{+ value : int}{}
		\umlVHVinherit[anchor1=north, anchor2=south, arm2=-15pt]
			{Type}{Symbol}
		\umlVHVinherit[anchor1=north, anchor2=south, arm2=-15pt]
			{NamedValue}{Symbol}
		\umlVHVinherit[anchor1=north, anchor2=south, arm2=-15pt]
			{Program}{Symbol}
		\umlVHVinherit[anchor1=north, anchor2=south, arm2=-15pt]
			{ReferenceType}{Type}
		\umlVHVinherit[anchor1=north, anchor2=south, arm2=-15pt]
			{MethodType}{Type}
		\umlVHVinherit[anchor1=north, anchor2=south, arm2=-15pt]
			{Array}{ReferenceType}
		\umlVHVinherit[anchor1=north, anchor2=south, arm2=-15pt]
			{Class}{ReferenceType}
		\umlVHVinherit[anchor1=north, anchor2=south, arm2=-30pt]
			{Constant}{NamedValue}
		\umlVHVinherit[anchor1=north, anchor2=south, arm2=-30pt]
			{Method}{NamedValue}
		\umlVHVuniaggreg[attr=methodType|1, anchor1=-120, anchor2=-60,arm2=-20pt,pos=1.9,align=left]
			{Method}{MethodType}
		\umlVHuniaggreg[attr=type|1,anchor1=-130, anchor2=-20,pos=1.9,align=left]
			{NamedValue}{Type}
	
	\end{umlpackage}
\end{tikzpicture} 
	\caption{Pojednostavljena struktura klasa koje predstavljaju simbole}
\end{figure}


\begin{figure}[h]

	\centering
	\begin{tikzpicture}
	\tikzumlset{font=\footnotesize}

	\begin{umlpackage}[x=0,y=0]{symbols}
	
		\umlclass[x=0pt,y=0pt]{Scope}{
			}{
				+ define(s: Symbol): void \\ 
				+ resolve(name: string): Symbol \\
				+ begin(): iterator \\
				+ end(): iterator
			}
		
		\umlemptyclass[x=-140pt,y=22pt]{Symbol}
		
		\umlclass[x=-86pt,y=-115pt]{MethodArguments}
			{}{
				+ matchArguments(args: vector)
			}

		\umlclass[x=180pt,y=-23pt]{SplitScope}
			{}{
				+ methodBegin(): iterator \\
				+ methodEnd(): iterator \\
				+ variableBegin(): iterator \\
				+ variableEnd(): iterator \\
				+ constantBegin(): iterator \\
				+ constantEnd(): iterator \\
				+ classBegin(): iterator \\
				+ classEnd(): iterator
			}
		
%		\umlemptyclass[x=-160pt,y=-50pt]{SplitScope}
		
		\umlemptyclass[x=60pt,y=-110pt]{ClassScope}
		
		\umlemptyclass[x=-140pt,y=-40pt]{Method}
		
		\umlemptyclass[x=80pt,y=-160pt]{Class}
		
		\umlemptyclass[x=120pt,y=-160pt]{Program}
		
		\umlemptyclass[x=210pt,y=-160pt]{GlobalScope}
		
	\end{umlpackage}
\end{tikzpicture}
	\caption{Pojednostavljena struktura klasa koje implementiraju i koriste opsege vidljivosti}
\end{figure}

\begin{figure}[h]

	\centering
	\begin{tikzpicture}
	\tikzumlset{font=\footnotesize}
	\begin{umlpackage}[x=0,y=0]{symbols}

		\umlclass[x=0pt,y=0pt]{Method}{}{}

		\umlclass[x=0pt,y=0pt]{MethodArguments}{}{}
		
		\umlclass[x=0pt,y=0pt]{MethodType}{
			
		}{
		}

	\end{umlpackage}
\end{tikzpicture}
	\caption{Pojednostavljena struktura klasa koje opisuju metod}
\end{figure}

\begin{figure}[h]

	\centering
	\begin{tikzpicture}
	\tikzumlset{font=\footnotesize}
	\begin{umlpackage}[x=0,y=0]{semantics}
		\umlclass[x=0pt,y=0pt]{Symbols}{}{}
	\end{umlpackage}
\end{tikzpicture}
	\caption{'Symbols' klasa predstavlja spoljni interfejs ka predstavljenim klasama}
\end{figure}

\section{Semantička analiza}


