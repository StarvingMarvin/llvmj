\chapter{Implementacija semantike}

Ovo poglavlje opisuje arhitekturalne odluke i klasnu hijerarhiju koja je ključna u semantičkoj analizi i implementaciji osobina jezika.
Način na koji se iz memorijskog modela generiše LLVM međureprezentacija je predmet narednog poglavlja.


\section{Simboli}

Centralno mesto u ovoj fazi MicroJava prevodioca zauzima hijerarhija klasa vezanih za simbole, opsege i tabelu simbola.

U MicroJavi možemo izdvojiti nekoliko vrsta simbola, kao što su primitivni tipovi, promenljive, konstante, klase i metode.
Radi što lakše analize, a kasnije i generisanja koda, ovi simboli predstavljeni su hijerarhijom klasa kao na dijagramu \ref{symbols}.

\begin{figure}[h]

	\centering
	\begin{tikzpicture}
	\tikzumlset{font=\footnotesize}

	\begin{umlpackage}[x=0,y=0]{symbols}

		\umlclass[x=210pt,y=250pt]{Symbol}
			{+ name : string }{} 

		\umlclass[x=90pt,y=165pt]{Type}
			{}{
				+ operator==(t: Type) : bool \\ 
				+ compatible(t: Type) : bool
			}

		\umlemptyclass[x=290pt,y=176pt]{NamedValue}

		\umlemptyclass[x=210pt,y=176pt]{Program}

		\umlemptyclass[x=40pt,y=85pt]{ReferenceType}

		\umlemptyclass[x=10pt,y=20pt]{Array}

		\umlemptyclass[x=70pt,y=20pt]{Class}

		\umlemptyclass[x=140pt,y=85pt]{MethodType}

		\umlemptyclass[x=250pt,y=85pt]{Method}

		\umlclass[x=330pt,y=80pt]{Constant}
			{+ value : int}{}

		\umlVHVinherit[anchor1=north, anchor2=south, arm2=-15pt]
			{Type}{Symbol}

		\umlVHVinherit[anchor1=north, anchor2=south, arm2=-15pt]
			{NamedValue}{Symbol}

		\umlVHVinherit[anchor1=north, anchor2=south, arm2=-15pt]
			{Program}{Symbol}

		\umlVHVinherit[anchor1=north, anchor2=south, arm2=-15pt]
			{ReferenceType}{Type}

		\umlVHVinherit[anchor1=north, anchor2=south, arm2=-15pt]
			{MethodType}{Type}

		\umlVHVinherit[anchor1=north, anchor2=south, arm2=-15pt]
			{Array}{ReferenceType}

		\umlVHVinherit[anchor1=north, anchor2=south, arm2=-15pt]
			{Class}{ReferenceType}

		\umlVHVinherit[anchor1=north, anchor2=south, arm2=-40pt]
			{Constant}{NamedValue}

		\umlVHVinherit[anchor1=north, anchor2=south, arm2=-40pt]
			{Method}{NamedValue}

		\umlVHVuniaggreg[attr=methodType|1, anchor1=-120, anchor2=-60,arm2=-20pt,pos=1.9,align=left]
			{Method}{MethodType}

		\umlVHuniaggreg[attr=type|1,anchor1=-130, anchor2=-16,pos=1.9,align=left]
			{NamedValue}{Type}
	
		\umlHVHuniaggreg[attr=returnType|1, anchor1=10,  anchor2=-16, arm2=40pt, pos=1.4]
			{MethodType}{Type}
		\umlHVHuniaggreg[attr=*|arguments, anchor1=-10,  anchor2=-16, arm2=55pt, pos=1.8]
			{MethodType}{Type}
	\end{umlpackage}
\end{tikzpicture} 
	\caption{Pojednostavljena struktura klasa koje predstavljaju simbole}
	\label{symbols}
\end{figure}

Prilikom analize programa, bitno je da za tipove vrednosti u nekom izrazu utvrdimo da li su jednaki očekivanom tipu, odnosno da li su kompatabilni u slučaju dodele.
Klasa \code{Type} deklariše virtualne metode \code{operator==} i \code{compatible}, a podrazumevana implementacija odgovara semantici primitivnih tipova: tipovi su jednaki ako se isto zovu, a kompatibilni su samo ako su jednaki.

Klasa \code{ReferenceType} prihvata \mj{null} kao compatibilan tip.

Tipovi nizova i metoda se implicitno registruju u globalnu tabelu simbola prilikom prvog nailaženja u kodu.



\begin{figure}[h]
	\centering
	\begin{tikzpicture}
	\tikzumlset{font=\footnotesize}

	\begin{umlpackage}[x=0,y=0]{semantics}
		\umlclass[x=0pt,y=0pt]{Scope}{+ name : string }{}
	\end{umlpackage}
\end{tikzpicture}
	\caption{Pojednostavljena struktura klasa koje implementiraju i koriste opsege vidljivosti}
\end{figure}

MicroJava metod je implementiran kao vrednost koja ima tip 

\begin{figure}[h]
	\centering
	\begin{tikzpicture}
	\tikzumlset{font=\footnotesize}
	\begin{umlpackage}[x=0,y=0]{symbols}

		\umlclass[x=0pt,y=0pt]{Method}{}{}

		\umlclass[x=0pt,y=0pt]{MethodArguments}{}{}
		
		\umlclass[x=0pt,y=0pt]{MethodType}{
			
		}{
		}

	\end{umlpackage}
\end{tikzpicture}
	\caption{Pojednostavljena struktura klasa koje opisuju metod}
\end{figure}

\begin{figure}[h]
	\centering
	\begin{tikzpicture}
	\tikzumlset{font=\footnotesize}
	\begin{umlpackage}[x=-2.5,y=0]{semantycs}
		\umlemptyclass[x=0, y=6]{SemanticNodeVisitor}
	\end{umlpackage}
	\begin{umlpackage}[x=2.5,y=0]{codegen}
		\umlemptyclass[x=0, y=6]{CodegenVisitor}
	\end{umlpackage}
	
	\begin{umlpackage}[x=0,y=0]{symbols}    
	    \umlclass[x=0pt,y=0pt]{Symbols}{
			- global: GlobalScope\\
			- scopes: stack$\langle$Scope$\rangle$
		}{
		    + defineArray(name: string, type: Type): void \\
            + defineConstant(name: string, t: Type, val: int): void \\
		    + defineNamedValue(name: string, t: Type): void \\
		    \\
		    + resolve(name: string): Symbol \\
            + resolveClass(name: string): Class \\
            + resolveMethod(name: string): Method \\
            + resolveNamedValue(name: string): NamedValue \\
            + resolveType(name: string): Type \\  
            \\
            + enterProgramScope(name: string): Program \\
            + enterClassScope(name: string):  Class \\
            + enterMethodArgumentsScope(): MethodArguments \\
            + enterMethodScope(name: string, returnType: Type, arguments: MethodArguments): Method \\
            
            + leaveScope(): void
		}
		\umlemptyclass[x=-2, y=-5]{Symbol}
	    \umlemptyclass[x=2, y=-5]{Scope}
	    
	    \umlVHVuniassoc[anchor1=center, anchor2=north]{Symbols}{Symbol}
	    \umlVHVuniassoc[anchor1=center, anchor2=north]{Symbols}{Scope}
	\end{umlpackage}
	\umlVHVuniassoc[anchor1=south, anchor2=north, weight=0.2]{SemanticNodeVisitor}{Symbols}
	\umlVHVuniassoc[anchor1=south, anchor2=north, weight=0.2]{CodegenVisitor}{Symbols}
\end{tikzpicture}
	\caption{'Symbols' klasa je fasada ka predstavljenim klasama}
\end{figure}

\section{Semantička analiza}

ASTWalker - utility klasa za prolazak kroz stablo koje je generisao antlr.

External visitor pattern

mnogo malih node visitor klasa

rezultat analize je tabela simbola

