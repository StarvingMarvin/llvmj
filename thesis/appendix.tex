
\appendix

\chapter{Specifikacija MicroJave}
\label{ch:microjava-spec}

Programski jezik MikroJava 
 
Ovaj dodatak opisuje programski jezik MikroJava koji se koristi u praktičnom delu kursa programskih 
prevodilaca (IR4PP1) na Elektrotehničkom fakultetu u Beogradu. MikroJava je slična Javi, ali je mnogo 
jednostavnija. 
 
 
\section{Opšte osobine jezika}
\begin{itemize}
\item MikroJava program se sastoji od jedne klase koja ima statička polja i statičke metode. Ne postoje druge 
spoljašnje klase, već samo unutrašnje koje se mogu koristiti kao tipovi podataka. 
\item Glavna metoda MikroJava programa se uvek zove main(). Kada se poziva MikroJava program 
izvršava se ta metoda. 
\item Postoje
\begin{itemize}
\item Celobrojne i znakovne konstante (int i char), ali ne i znakovni nizovi (string). 
\item Promenljive: sve promenljive predstavljaju reference; promenljive u glavnoj klasi su statičke. 
\item Osnovni tipovi: int, char (Ascii). 
\item Strukturirani tipovi: jednodim. nizovi kao u Javi, unutrašnje klase sa poljima, ali bez metoda. 
\item Statičke metode u glavnoj klasi. 
\end{itemize}
\item Ne postoji garbage kolektor (alocirani objekti se samo dealociraju nakon kraja programa). 
\item Predeklarisane procedure su ord, chr, len. 
 
\end{itemize}
 
 
 
\subsection*{Primer programa}

\begin{lstlisting}
class P

	final int size = 10;

	class Table {
		int pos[];
		int neg[];
	}

	Table val;
{
	void main()
		int x, i;
	{ // Initialize val
		val = new Table;
		val.pos = new int[size];
		val.neg = new int[size];
		i = 0;
		while (i < size) {
			val.pos[i] = 0; val.neg[i] = 0;
			i++;
		}
        // Read values
		read(x);
		while (x > 0) {
			if (0 <= x && x < size) {
				val.pos[x]++;
			} else if (-size < x && x < 0) {
				val.neg[-x]++;
			}
			read(x);
		}
	}
}
\end{lstlisting}

\section{Sintaksa}
\begin{tabular}{l l}
Program & = ``class" ident \{ ConstDecl $\mid$ VarDecl $\mid$ ClassDecl \} ``\{" \{ MethodDecl \} ``\}". \\
ConstDecl & = ``final" Type ident ``=" (number $\mid$ charConst) ``;". \\
VarDecl & = Type ident [``[" ``]"] \{``," ident [``[" ``]"]\} ``;". \\
ClassDecl & = ``class" ident ``{" {VarDecl} ``}". \\
MethodDecl & = (Type $\mid$ ``void") ident ``(" [FormPars] ``)" \{VarDecl\} ``\{" \{Statement\} ``\}". \\
FormPars & = Type ident [``[" ``]"] \{``," Type ident [``[" ``]"]\}. \\
Type & = ident. \\

Statement & = Designator (``=" Expr $\mid$ ``(" [ActPars] ``)" $\mid$ ``++" $\mid$ ``--") ``;" \\
& $\mid$ ``if"{} ``(\grqq{} Condition ``)\grqq{} Statement [``else\grqq{} Statement] \\
& $\mid$ ``while"{} ``(\grqq Condition ``)\grqq{} Statement \\
& $\mid$ ``break"{} ``;" \\
& $\mid$ ``return"{} [Expr] ``;" \\
& $\mid$ ``read"{} ``(\grqq{} Designator ``)"{} ``;" \\
& $\mid$ ``print"{} ``(" Expr [``," number] ``)"{} ``;" \\
& $\mid$ ``\{" \{Statement\} ``\}". \\

ActPars & = Expr \{``," Expr\}. \\
Condition & = CondTerm \{``$\mid \mid$\grqq{} CondTerm\}. \\
CondTerm & = CondFact \{``\&\&\grqq{} CondFact\}. \\
CondFact & = Expr Relop Expr. \\
Expr & = [``-"] Term \{Addop Term\}. \\
Term & = Factor \{Mulop Factor\}. \\

Factor & = Designator [``(" [ActPars] ``)"] \\
& $\mid$ number \\
& $\mid$ charConst \\
& $\mid$ ``new" Type [``[" Expr ``]"] \\
& $\mid$ ``(" Expr ``)". \\

Designator & = ident \{ ``." ident $\mid$ ``[" Expr ``]"\}. \\
Relop & = ``==" $\mid$ ``!=" $\mid$ ``>" $\mid$ ``>=" $\mid$ ``<" $\mid$ ``<=". \\
Addop & = ``+" $\mid$ ``-". \\
Mulop & = ``*" $\mid$ ``/" $\mid$ ``\%". \\

\end{tabular}

\subsection*{Leksičke strukture}
 
\begin{tabular}{l l}

Ključne reči: & break, class, else, final, if, new, print, read, return, void, while \\ 
Vrste tokena: & ident = letter \{letter $\mid$ digit $\mid$ ``\_"\}.  \\ 
& number = digit {digit}.  \\
& charConst = ``'" printableChar ``'".  \\
Operatori: & +, -, *, /, \%, ==, !=, >, >=, <, <=, \&\&, $\mid \mid$, =, ++, --, ;, zarez, ., (, ), [, ], {, }  \\ 
Komentari: & // do kraja linije 
\end{tabular}

 
\section{Semantika}
 
Svi pojmovi u ovom dokumentu, koji imaju definiciju, su podvučeni da bi se naglasilo njihovo posebno 
značenje. Definicije tih pojmova su date u nastavku. 

\hspace{0cm} \\

\underline{Tip reference}

Nizovi i klase su tipa reference. 

\hspace{0cm} \\

\underline{Tip konstante}

\begin{itemize}
\item Tip celobrojne konstante (npr. 17) je int. 
\item Tip znakovne konstante (npr. 'x') je char. 
\end{itemize}

\hspace{0cm} \\
 
\underline{Ekvivalentni tipovi podataka}

Dva tipa podataka su ekvivalentna

\begin{itemize}
\item ako imaju isto ime, ili 
\item ako su oba nizovi, a tipovi njihovih elemenata su \underline{ekvivalentni}. 
\end{itemize}
 
\hspace{0cm} \\

\underline{Kompatibilni tipova podataka}

Dva tipa podataka su kompatibilna 

\begin{itemize}
\item ako su \underline{ekvivalentni}, ili 
\item ako je jedan od njih \underline{tip reference}, a drugi je tipa \textit{null}. 
\end{itemize}
 
Tip \textit{src} je kompatibilan pri dodeli sa tipom \textit{dst}

\hspace{0pt} \\

\underline{Kompatibilnost tipova podataka pri dodeli}

\begin{itemize}
\item ako su \textit{src} i \textit{dst} \underline{ekvivalentni}, ili
\item ako je \textit{dst} \underline{tip reference}, a \textit{src} je tipa \textit{null}. 
\end{itemize}

\hspace{0pt} \\

\underline{Predeklarisana imena}

\begin{tabular}{l p{10cm}}
int & tip svih celobrojnih vrednosti \\
char & tip svih znakovnih vrednosti \\
null & null vrednost promenljive tipa klase ili niza simbolički označava pokazivač koji ne pokazuje ni na jedan podatak \\
eol & kraj reda karaktera (odgovara znaku `\textbackslash n'); print(eol) vrši prelazak u novi red \\
chr & standardna metoda; chr(i) vrši konverziju celobrojnog izraza i u karakter (char) \\
ord & standardna metoda; ord(ch) vrši konverziju karaktera ch u celobrojnu vrednost (int) \\ 
len & standardna metoda; len(a) vraća broj elemenata u nizu a \\

\end{tabular}

\hspace{0pt} \\

\underline{Opseg važenja}

Opseg važenja (scope) predstavlja tekstualni doseg metode ili klase. Prostire se od početka definicije 
metode ili klase do zatvorene velike zagrade na kraju te definicije. Opseg važenja ne uključuje imena 
koja su deklarisana u opsezima koji su leksički ugnježdeni unutar njega. U opsegu se “vide” imena 
deklarisana unutar njega i svih njemu spoljašnjih opsega. Pretpostavka je da postoji veštački globalni 
opseg (universe), za koji je glavna klasa (koja predstavlja i glavni program) lokalna i koji sadrži sva 
predeklarisana imena.  

Deklaracija imena u unutrašnjem opsegu $S$ sakriva deklaraciju istog imena u spoljašnjem opsegu. 
 
\subsection*{Napomena}
 
\begin{itemize}
\item Indirektna rekurzija nije dozvoljena i svako ime mora biti deklarisano pre prvog korišćenja.  
\item Predeklarisana imena (npr. int ili char) mogu biti redeklarisani u unutrašnjem opsegu (ali to nije 
preporučljivo).  
\end{itemize} 
 
\section{Kontekstni uslovi}
 
\subsection*{Opšti kontekstni uslovi}

\begin{itemize}
\item Svako ime u programu mora biti deklarisano pre prvog korišćenja. 
\item Ime ne sme biti deklarisano više puta unutar istog opsega. 
\item U programu mora postojati metoda sa imenom main. Ona mora biti deklarisana kao void metoda bez 
argumenata. 
\end{itemize}

\subsection*{Kontekstni uslovi za standardne metode}

\begin{tabular}{l l}
chr(e) & \textit{e} mora biti izraz tipa int. \\
ord(c) & \textit{c} mora biti tipa char. \\
len(a) & \textit{a} mora biti niz.
\end{tabular}
 
\subsection*{Kontekstni uslovi za MikroJava smene}
 
 
\textbf{Program = ``class" ident \{ConstDecl $\mid$ VarDecl $\mid$ ClassDecl\} ``\{" \{MethodDecl\} ``\}". }

\hRule \\[0.2cm]

\textbf{ConstDecl = ``final" Type ident ``=" (number $\mid$ charConst) ``;".}

• Tip terminala \textit{number} ili \textit{charConst} mora biti \underline{ekvivalentan} tipu Type. 

\hRule \\[0.2cm]
 
\textbf{VarDecl = Type ident [``[" ``]"] \{``," ident [``[" ``]"]\} ``;".} 

\hRule \\[0.2cm]

\textbf{ClassDecl = ``class" ident ``\{" \{VarDecl\} ``\}".} 

\hRule \\[0.2cm]
 
\textbf{MethodDecl = (Type $\mid$ ``void") ident ``(" [FormPars] ``)" \{VarDecl\} ``\{" \{Statement\} ``\}".} 

• Ako metoda nije tipa void mora imati iskaz return unutar svog tela (ovo se proverava pri izvršenju). 

\hRule \\[0.2cm]
 
\textbf{FormPars = Type ident [``["{} ``]"] \{``," Type ident [``["{} ``]"]\}.} 

\hRule \\[0.2cm]

\textbf{Type = ident.} 

• \textit{ident} mora označavati tip podataka. 
 
\hRule \\[0.2cm] 
  
\textbf{Statement = Designator ``=" Expr ``;".} 

• \textit{Designator} mora označavati promenljivu, element niza ili polje unutar objekta. 

• Tip neterminala Expr mora biti kompatibilan pri dodeli sa tipom neterminala Designator. 
 
\textbf{Statement = Designator (``++" $\mid$ ``--") ``;".} 

• \textit{Designator} mora označavati promenljivu, element niza ili polje unutar objekta. 

• \textit{Designator} mora biti tipa int. 
 
\textbf{Statement = Designator ``(" [ActPars] ``)"{} ``;".} 

• \textit{Designator} mora označavati metodu. 
 
\textbf{Statement = ``break".} 

• Iskaz break se može koristiti samo unutar while ciklusa. 
 
\textbf{Statement = ``read"{} ``("{} Designator ``)"{} ``;".} 

• \textit{Designator} mora označavati promenljivu, element niza ili polje unutar objekta. 

• \textit{Designator} mora biti tipa int ili char. 
 
\textbf{Statement = ``print"{} ``("{} Expr [``," number] ``)"{} ``;".} 

• \textit{Expr} mora biti tipa int ili char. 
  
\textbf{Statement = ``return" [Expr] .} 

• Tip neterminala \textit{Expr} mora biti isti kao i \underline{povratni tip} tekuće metode.

• Ako neterminal \textit{Expr} nedostaje, tekuća metoda mora biti deklarisana kao void. 

\begin{tabular}{l l}
\textbf{Statement} & \textbf{= ``if" ``("{} Condition ``)"{} Statement [``else"{} Statement]}\\
& \textbf{$\mid$ ``while" ``("{} Condition ``)"{} Statement} \\
& \textbf{$\mid$ ``\{" {Statement} ``\}"} \\
& \textbf{$\mid$ ";".} 
\end{tabular}

\hRule \\[0.2cm]

\textbf{ActPars = Expr \{``," Expr\}. }

• Broj formalnih i stvarnih argumenata metode mora biti isti. 

• Tip svakog stvarnog argumenta mora biti \underline{kompatibilan pri dodeli} sa tipom svakog formalnog 
argumenta na odgovarajućoj poziciji. 

\hRule \\[0.2cm]

\textbf{Condition = CondTerm {``$\mid\mid$"{} CondTerm}.} 

\hRule \\[0.2cm]

\textbf{CondTerm = CondFact {``\&\&"{} CondFact}.} 

\hRule \\[0.2cm]
 
\textbf{CondFact = Expr Relop Expr.} 

• Tipovi oba izraza moraju biti \underline{kompatibilni}.
 
• Uz promenljive tipa klase ili niza, od relacionih operatora, mogu se koristiti samo != i ==. 

\hRule \\[0.2cm]
 
\textbf{Expr = Term.} 
 
\textbf{Expr = ``-"Term.}

• \textit{Term} mora biti tipa int. 
 
\textbf{Expr = Expr Addop Term.}

• \textit{Expr} i \textit{Term} moraju biti tipa int. 

\hRule \\[0.2cm]

\textbf{Term = Factor.} 
 
\textbf{Term = Term Mulop Factor.}

• \textit{Term} i \textit{Factor} moraju biti tipa int. 
 
\hRule \\[0.2cm]
 
\textbf{Factor = Designator $\mid$ number $\mid$ charConst| ``(" Expr ``)".} 
 
\textbf{Factor = Designator ``(" [ActPars] ``)".}

• \textit{Designator} mora označavati metodu. 
 
\textbf{Factor = ``new" Type .}

• \textit{Type} mora označavati klasu. 
 
\textbf{Factor = ``new" Type ``[" Expr ``]".}

• Tip neterminala \textit{Expr} mora biti int. 

\hRule \\[0.2cm]

\textbf{Designator = Designator ``." ident .}

• Tip neterminala \textit{Designator} mora biti klasa. 

• \textit{Ident} mora biti polje neterminala \textit{Designator}. 
 
\textbf{Designator = Designator ``[" Expr ``]".}

• Tip neterminala \textit{Designator} mora biti niz.

• Tip neterminala \textit{Expr} mora biti int. 
 
\hRule \\[0.2cm]

\textbf{Relop = ``==" $\mid$ "!=" $\mid$ ">" $\mid$ ">=" $\mid$ "<" $\mid$ "<=".} 

\hRule \\[0.2cm] 
 
\textbf{Addop = ``+" $\mid$ ``-".} 

\hRule \\[0.2cm]

\textbf{Mulop = ``*" $\mid$ ``/" $\mid$ ``\%".} 


\section{Implementaciona ograničenja}
 
• Ne sme se koristiti više od 256 lokalnih promenljivih. 

• Ne sme se koristiti više od 65536 globalnih promenljivih. 

• Klasa ne sme imati više od 65536 polja. 

• Izvorni kod programa ne sme biti veći od 8 KB. 


\chapter{Listing gramatike}

\lstinputlisting{../src/parser/MicroJava.g}

\chapter{Primer programa i sintaksnog stabla}
\label{ch:example}

Primer programa koji sortira listu unetih brojeva:

\lstinputlisting{../example/example.mj}

Generisano sintaksno stablo:

\lstinputlisting{../example/example.ast}


\chapter{LLVM}
\label{ch:llvm}
Za inspekciju i razumevanje generisanog koda, korisno je poznavati osnove LLVM asemblerskog jezika koji je ovde ukratko predstavljen.

\section{LLVM tipovi}

Tipovi podataka u LLVM-u omogućavaju da brojne optimizacije budu spovedene direktno nad međureprezentacijom, 
bez potrebe obavljanja uporedne analize mogućih vrednosti u nekom registru pre primene transformacije.
LLVM je striktan po pitanju tipova u smislu da ni u jednom trenutku neće izvršiti implicitnu konverziju između dva tipa, 
uključujući tu i proširenje brojnih vrednosi ili pretvaranje jednog tipa pokazivača u drugi. 
Sa druge strane korisnika ništa ne sprečava da izvrši konverziju između bilo koja dva tipa odgovarajućim instrukcijama.

Tipovi se mogu podeliti na primitivne kao što su tipovi celobrojnih i realnih brojeva, labela, i metapodatak; 
i izvedene tipove kao što su nizovi, funkcije, pokazivači, strukture i vektori.

Osim ove podele bitno je istaći da nisu svi tipovi ravnopravni.
Oni tipovi koji mogu biti rezultat LLVM instrukcija i mogu se skladištiti u memoriji, prosleđivati kao parametar ili vraćati kao rezultat funkcije, nazivaju se tipovi prvog reda. Tipovi prvog reda su: celobrojni i realni tipovi, pokazivači, nizovi, strukture, vektori, labele i metapodaci.


\subsection*{Primitivni tipovi}

Primitivni tipovi su osnovni gradivni blokovi LLVM sistema.

Pošto tipovi celih i realnih brojeva zavise od podrške konkretne procesorske arhitekture,
sam LLVM pruža veliku fleksibilnost u njihovom deklarisanju
pa su mogući tipovi bilo koje širine od 1 do $2^{23} - 1$ (pribliižno 8 miliona) koji se označavaju sa \code{iN} gde \code{N} predstavlja širinu u bitima.

Postoji 5 tipova realnih brojeva: 
\code{float} koji je širine 32 bita, 
\code{double} širine 64 bita, 
\code{fp128} od 128 bita sa mantisom od 112 bita, 
kao i hardverski specifične x86\_{}fp80 širine 80 bita 
i ppc\_{}fp128 koji takodje ima 128 bita, ali mantisu širine 64 bita.


\section{Korišćene LLVM instrukcije}

Sledi pregled svih potrebnih instrukcija za razumevanje generisanog MicroJava koda.

\subsection*{Memorijske instrukcije}

\begin{description}
\item[\code{alloca}] Alokacija memorije na steku. Memorija se automatski oslobađa po izlasku iz funkcije.
\item[\code{load}] Učitavanje vrednosti sa date memorijske lokacije u registar. Moguće je samo učitavanje vrednosti prvog reda.
\item[\code{store}] upisivanje u memoriju
\item[\code{gep}] Računanje vrednosti pokazivača u okviru niza ili strukture.
\end{description}

\subsection*{Aritmetičko logičke instrukcije}

\begin{description}
\item[\code{add, sub, mul, sdiv, srem, neg}] Celobrojne aritmetičke operacije. Oba operanda moraju biti istog tipa.
\item[\code{icmp}] Celobrojno poređenje datih argumenata. Prvi argument je vrsta poređenja i moze biti
jedno od: \code{eq} - jednako,  \code{ne} - različito, \code{sgt} - veće, \code{sge} - veće ili jednako,  \code{slt} - manje i  \code{sle} - manje ili jednako, 
\item[\code{and, or}] Bit-logičke operacije. Argumenti mogu biti bilo kog celobrojnog tipa, ali u slučaju MicroJava-e, uvek je u pitanju \code{i1}
\end{description}

\subsection*{Kontrola toka}

\begin{description}
\item[\code{br}] Instrukcija skoka kojom se kontrola toka prebacuje bloku unutar iste funkcije. Postoje dva oblika ove instrukcije: \textit{bezuslovni skok} koji za parametar prima labelu bloka na koji se skače i \textit{uslovni skok} koji prima vrednost tipa \code{i1} i dve labele.
\item[\code{call}] poziv funkcije
\item[\code{ret}] Vraća kontrolu toka, a opciono i povratnu vrednost pozivaocu trenutne funkcije. Da bi funkcija bila ispravna, tip vrednosti koja se vraca mora biti jednak povratnom tipu finkcije, odnosno vrednost mora biti \code{void} ako funkcija ne vraća vrednost. Povratni tip mora biti prvog reda.
\end{description}

\subsection*{Konverzije}

\begin{description}
\item[\code{bitcast}] Konvertuje datu vrednost u zadati tip bez promene bitova. Pogodno za konvertovanje tipova pokazivača ili tip pokazivača u celobrojni tip odgovarajuće širine.
\item[\code{sextorbitcast}]
\item[\code{zext}] Uzima kao parametre celobrojnu vrednost tipa \code{t1} i celobrojni tip \code{t2} veće preciznosti, proširi datu vrednost nulama sa leve strane i vrati vrednost tipa \code{t2}. Ova operacija uvek vrši konverziju, zato sto širina tipa \code{t2} mora biti strogo veća od širine tipa \code{t1}
\item[\code{trunc}] Uzima kao parametre celobrojnu vrednost tipa \code{t1} i celobrojni tip \code{t2} manje preciznosti, odbaci više bite date vrednosti i vrati vrednost tipa \code{t2}. Ova operacija uvek vrši konverziju, zato što širina tipa \code{t2} mora biti strogo manja od širine tipa \code{t1}
\end{description}



\chapter{Dokumentacija MicroJava kompajlera}

\section{Instalacija}

Projekat se može instalirati kompajliranjem iz koda. Kod je razvijan i testiran
na Linuxu, ali bi trebalo da radi i na drugim platformama.

\subsection*{Zavisnosti}

Alati potrebni za izgradnju projekta:
\begin{itemize}
\item CMake
\item Antlr3
\item Odgovarajuće platformsko razvojno okruženje (Make/GCC, VisualStudio, XCode)
\end{itemize}

Biblioteke koje treba da su dostupne na sistemu:
\begin{itemize}
\item LLVM 3.6
\item Antlr3c
\end{itemize}

Na Debian Linuxu, sve potrebne zavisnosti mogu da se instaliraju na sledeći način:

\begin{lstlisting}
# apt-get install build-essential cmake antlr3 antlr3c-dev \
          llvm-3.6 llvm-3.6-dev llvm-3.6-tools
\end{lstlisting}

\subsection*{Kompajliranje}

Ukoliko su sve zavisnosti zadovoljene, pokretanje cmake alata će da napravi
odgovarajuće projektne fajlove za trenutnu platformu.

Na Linuxu će biti generisan `Makefile`, pa kompajliranje projeka izgleda ovako:

\begin{lstlisting}
$ cmake
$ make
\end{lstlisting}

Izvršni fajlovi se nalaze u \texttt{src/mjc} diretkorijumu.

\section{Korišćenje}

Interpreter se pokreće komandom \texttt{mji} koja od argumenata prima putanju do fajla koji treba izvršiti.

Kompajler se pokreće komandom \texttt{mjc} koja osim putanje do fajla opciono prima još dva argumenta:
\begin{description}
\item[\texttt{-O}] nivo optimizacije, koji može biti vrednost 0-3 (podrazumevano 2)
\item[\texttt{-t}] tip izlaza koji može biti:

\begin{itemize}
    \item \texttt{ast} - tekstualna reprezentacija apstraktnog sintaksnog stabla
    \item \texttt{llvm} - LLVM asembler
    \item \texttt{bc} - LLVM bajt kod
    \item \texttt{asm} - platformski asembler
    \item \texttt{obj} - objektni fajl (podrazumevana vrednost)
\end{itemize}
\end{description}

Da bi se dobio izvršni fajl, objektni fajl je potrebno linkovati, na primer na sledeći način:

\begin{lstlisting}
$ gcc -o program program.o
\end{lstlisting}
