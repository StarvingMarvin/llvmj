\chapter{Uvod}

Posao prevođenja programskog jezika u mašinsku reprezentaciju se može jako jednostavno podeliti na dva dela: analiza ulaznog programa i generisanje krajnjeg izlaza\cite{dragon}. 
Ako se pogleda malo detaljnije, vidi se da analizu sačinjava nekoliko koraka. 
Svaki korak transformiše svoj ulaz u oblik pogodniji za dalji proces prevođenja. 
Na samom početku \emph{leksičkom analizom} će se niz slova i znakova sa ulaza pretvoriti u niz \emph{tokena}, logičkih celina kao što su idetifikator, ključna reč, broj i slično. 
Potom će \emph{sintaksnom analizom} za niz tokena da se utvrdi da li odgovaraju gramatičkim pravilima jezika. 
Potrebno je sakupiti simbole koji se u programu javljaju i izgraditi neku strukturnu model programa u memoriji pogodnu za dalju obradu. 
Ako program zadovoljava gramatiku jezika, to i dalje ne mora da znači da je ispravan. 
Slično govornim jzicima, gramatički validna rečenica ne mora da ima smisla, pa je potrebno proći kroz izgrađeni model programa, i proveriti da li su zadovoljena i semantička pravila. 
Sada je moguće preći i na generisanje koda.

Slično analizi i generisanje se sastoji iz mnoštva manjih koraka. 
Model programa koji je bio pogodan za semantičku analizu treba dalje transformisati u novi model koji je pogodniji za optimizaciju i generisanje koda. 
Neke tehnike optimizacije su primenjive nezavisno od konkretne procesorske arhitekture i njima je moguće transformisati trenutni model programa. 
Zatim se generišu mašinske instrukcije i primenjuju optimizacije specifične za ciljanu procesorsku arhitekturu.

Postoji mnoštvo parser-generatora koji se razlikuju po načinu zadavanja gramatike, strategije parsiranja i jezika u kome će željeni parser biti implementiran.
ANTLR se između ostalog izdvaja po tome što je osim opisivanja sintakse jezika koji želimo da prepoznamo moguće opisati i apstraktno sintaksno stablo koje odgovara prepoznatom gramatičkom pravilu. 
Na taj način ANTLR pokriva veći deo prve celine programskog prevodioca: leksičku i sintaksnu analizu i generisanje sintaksnog stabla.

Nasuprot generatorima parsera, ne postoji mnogo projekata koji omogućavaju da se na jednostavan način od reprezentacije programa u memoriji dodje do izgenerisanog mašinskog koda.

%% FIXME
llvm ide od tri address koda do mašinskih instrukcija.
nema dosta projekata kao sto je llvm - eventualno jvm i clr.

Između analize ulaza i proizvedenog mašinskog koda, leži semantika jezika, ono što %% FIXME


Pre detaljnijeg opisa implementacije pojedinih faza MicroJava prevodioca sledi kratak pregled samog jezika. 