\chapter{Uvod}

Posao prevođenja programskog jezika u mašinsku reprezentaciju se jednostavno može podeliti na dva dela: analizu ulaznog programa i generisanje krajnjeg izlaza\cite{dragon}. 
Ako se pogleda malo detaljnije, vidi se da analizu sačinjava nekoliko koraka. 
Svaki korak transformiše svoj ulaz u oblik pogodniji za dalji proces prevođenja. 
Na samom početku \emph{leksičkom analizom} će se niz slova i znakova sa ulaza pretvoriti u niz \emph{tokena}, logičkih celina kao što su idetifikator, ključna reč, broj i slično. 
Potom će se \emph{sintaksnom analizom} za niz tokena utvrditi da li odgovaraju gramatičkim pravilima jezika. 
Potrebno je sakupiti simbole koji se u programu javljaju (tabela simbola) i izgraditi strukturni model programa u memoriji pogodan za dalju obradu (apstraktno sintaksno stablo).
Ako program zadovoljava gramatiku jezika, to i dalje ne mora da znači da je ispravan.
Slično govornim jezicima, gramatički validna rečenica ne mora da ima smisla.
Stoga je potrebno proći kroz izgrađeni model programa i proveriti da li su zadovoljena i semantička pravila. 
Kada se utvrdi da program zadovoljava i semantička pravila moguće je preći na generisanje koda.

Slično analizi i generisanje se sastoji iz mnoštva manjih koraka.
Model programa koji je bio pogodan za semantičku analizu treba dalje transformisati u novi model koji je pogodniji za optimizaciju i generisanje koda, najčešće u obliku troadresnog koda.
Neke tehnike optimizacije su primenljive nezavisno od konkretne procesorske arhitekture i njima je moguće transformisati trenutni model programa.
Zatim se generišu mašinske instrukcije i primenjuju optimizacije specifične za ciljanu procesorsku arhitekturu.

ANTLR je parser generator i brine se o tumačenju ulaznog programa.
Ono što ANTLR izdvaja od drugih parser generatora je to što ide korak dalje i pomaže u generisanju apstraktnog sintaksnog stabla.

Za razliku od parser generatora, ne postoji mnogo projekata koji se bave optimizacijom i generisanjem koda nezavisno od izvornog jezika.
Iz tog razloga većina prevodilaca i interpretera svaki ponovo implemetira ove funkcionalnosti. 
Uz LLVM, tri platforme koje donekle predstavljaju izuzetak su 
JVM\footnote{\skr \eng Java Virtual Machine \ndash Java virtualna mašina}, 
CLR\footnote{\skr \eng Common Language Runtime \ndash Virtualna mašina koja stoji iza Majkrosoftove .net platforme} 
i GCC\footnote{\skr \eng GNU Compiler Collection \ndash Kolekcija kompajlera koju čine jezici C, C++, ObjectiveC, Ada i Fortran}. 
JVM i CLR predstavljaju virtualne mašine visokog nivoa i dolaze sa svojom semantikom vezanom za automatsko rukovanje memorijom, klasama, objektima i izuzecima. 
Iako su mnogi jezici napisani ili portovani na ove platforme, taj posao je bitno lakši za jezike koji se u većoj meri poklapaju sa Javom odnosno C\#-om. 
Ove platforme i sa strane generisanja koda imaju svoja ograničenja.
One ne omogućavaju da se za njih jednostavno dopiše generator koda za novu procesorsku arhitekturu ili nova optimizaciona tehnika. 
Ono što je po strukturi i svrsi približnije LLVM-u je GCC.
GCC sa svoje ulazne strane podržava nekoliko jezika, a sa izlazne može da generise kod za mnoge procesorske arhitekture.
U teoriji bi bilo moguće napisati podršku za novi jezik, nove procesore ili nove optimizacije.
U praksi se pokazalo da je unutrašnja arhitektura GCC-a suviše kompleksna i da su komponente prevodioca suviše međusobno povezane. 
Posledica toga je da trenutno ne postoji mnogo projekata nezavisnih od GCC-a koji koriste njegovu infrastrukturu.

Ono što preostaje u sredini je sama suština jezika: kako se jezik ponaša i na koji način implementira svoje osobine.
Sintaksa se prva uoči u toku upoznavanja nekog programskog jezika, pa joj se često prilikom poređenja daje preveliki značaj.
Semantika, iako nije očigledna, čini jezik i njegove karakteristike.
Na primer, Java i C imaju istu sintaksu za pristup elementu niza \ndash \code{a[i]}, ali je semantika različita.
Java će baciti izuzetak ukoliko indeks izlazi iz okvira niza, dok će C vratiti sadržaj tražene adrese, što može izazvati različite neželjene posledice.
Ova razlika u ponašanju nije primetna gledanjem u kod i zahteva dublje poznavanje jezika od same sintakse.
Zbog toga bih pored prepoznavanja ulaza i generisanja mašinskog koda izdvojio i detalje implementacije semantičkih odlika jezika kao bitnu celinu u opisu rada prevodioca.

Ove tri celine, koje čine programske prevodioce, opisane su u poglavljima prepoznavanje jezika, implementacija semantike i generisanje koda.
Naredno polavlje opisuje MicroJavu, jezik koji se implementira u okviru ovog rada.
