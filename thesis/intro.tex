\chapter{Uvod}

Posao prevođenja programskog jezika u mašinsku reprezentaciju se može jako jednostavno podeliti na dva dela: analiza ulaznog programa i generisanje krajnjeg izlaza\cite{dragon}. 
Ako se pogleda malo detaljnije, vidi se da analizu sačinjava nekoliko koraka. 
Svaki korak transformiše svoj ulaz u oblik pogodniji za dalji proces prevođenja. 
Na samom početku \emph{leksičkom analizom} će se niz slova i znakova sa ulaza pretvoriti u niz \emph{tokena}, logičkih celina kao što su idetifikator, ključna reč, broj i slično. 
Potom će \emph{sintaksnom analizom} za niz tokena da se utvrdi da li odgovaraju gramatičkim pravilima jezika. 
Potrebno je sakupiti simbole koji se u programu javljaju i izgraditi neki strukturni model programa u memoriji pogodnu za dalju obradu. 
Ako program zadovoljava gramatiku jezika, to i dalje ne mora da znači da je ispravan. 
Slično govornim jzicima, gramatički validna rečenica ne mora da ima smisla, pa je potrebno proći kroz izgrađeni model programa, i proveriti da li su zadovoljena i semantička pravila. 
Sada je moguće preći i na generisanje koda.

Slično analizi i generisanje se sastoji iz mnoštva manjih koraka. 
Model programa koji je bio pogodan za semantičku analizu treba dalje transformisati u novi model koji je pogodniji za optimizaciju i generisanje koda. 
Neke tehnike optimizacije su primenjive nezavisno od konkretne procesorske arhitekture i njima je moguće transformisati trenutni model programa. 
Zatim se generišu mašinske instrukcije i primenjuju optimizacije specifične za ciljanu procesorsku arhitekturu.

ANTLR je parser generator i brine se o tumačenju ulaznog programa. Ono što ANTLR izdvaja od drugih parser generatora je to što ide korak dalje i pomaže u generisanju apstraktnog sintaksnog stabla.

Za razliku od parser generatora, ne postoji previše projekata koji se bave optimizacijom i generisanjem koda nezavisno od izvornog jezika, pa tako većina prevodilaca i interpretera svaki ponovo implemetira ove funkcionalnosti. 
Tri platforme koje donekle predstavljaju izuzetak su JVM\footnote{\skr \eng Java Virtual Machine}, CLR\footnote{\skr \eng Common Language Runtime} i GCC\footnote{\skr \eng GNU Compiler Collection}. % ali nisu tolko strava

Ono što preostaje u sredini je sama suština jezika: kako se jezik ponaša i na koji način implementira svoje osobine. 
Java i C imaju istu sintaksu za pristup elementu niza \textmdash \code{a[i]}, ali različitu semantiku: 
Java će baciti izuzetak ukoliko indeks izlazi iz okvira niza, dok će C vratiti sadržaj tražene adrese, što može izazvati različite neželjene posledice.

Ostatak rada podeljen je u četiri poglavlja.
Nakon polavlja o MicroJavi koji opisuje jezik koji se implementira, slede poglavlja o svakoj od faza implementacije jezika: prepoznavanje ulaza, semantička analiza i generisanje koda.
