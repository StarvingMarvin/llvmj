\chapter{Uvod}

Posao prevođenja programskog jezika u mašinsku reprezentaciju se može jako jednostavno podeliti na dva dela: analiza ulaznog programa i generisanje krajnjeg izlaza\cite{dragon}. 
Ako se pogleda malo detaljnije, vidi se da analizu sačinjava nekoliko koraka. 
Svaki korak transformiše svoj ulaz u oblik pogodniji za dalji proces prevođenja. 
Na samom početku \emph{leksičkom analizom} će se niz slova i znakova sa ulaza pretvoriti u niz \emph{tokena}, logičkih celina kao što su idetifikator, ključna reč, broj i slično. 
Potom će \emph{sintaksnom analizom} za niz tokena da se utvrdi da li odgovaraju gramatičkim pravilima jezika. 
Potrebno je sakupiti simbole koji se u programu javljaju i izgraditi neku strukturnu model programa u memoriji pogodnu za dalju obradu. 
Ako program zadovoljava gramatiku jezika, to i dalje ne mora da znači da je ispravan. 
Slično govornim jzicima, gramatički validna rečenica ne mora da ima smisla, pa je potrebno proći kroz izgrađeni model programa, i proveriti da li su zadovoljena i semantička pravila. 
Sada je moguće preći i na generisanje koda.

Slično analizi i generisanje se sastoji iz mnoštva manjih koraka. 
Model programa koji je bio pogodan za semantičku analizu treba dalje transformisati u novi model koji je pogodniji za optimizaciju i generisanje koda. 
Neke tehnike optimizacije su primenjive nezavisno od konkretne procesorske arhitekture i njima je moguće transformisati trenutni model programa. 
Zatim se generišu mašinske instrukcije i primenjuju optimizacije specifične za ciljanu procesorsku arhitekturu.

antlr ide od lexera do asta.
ima dosta parsera.

llvm ide od tri address koda do mašinskih instrukcija.
nema dosta projekata kao sto je llvm - eventualno jvm i clr.

ostaje semantika u sredini.

naredno poglavlje opisuje mikrojava jezik, zatim deo koji rešava antlr, semantika i llvm.
